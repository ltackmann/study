\chapter{Actuarial mathematics}
Until the beginning of the 1980s, the concept of a pension was largely synonymous with the idea of a defined benefit (DB) plan; that is, the promise of a pension meant that upon retirement, an individual would receive a predetermined amount for the rest of his or her life.

Beginning in the 1980s, there was substantial growth in defined contribution (DC) plans. In these plans, the employer does not specify what benefits the retiree will receive. Instead, an investment account is opened for an employee, the employer and/or the employee make contributions to it, and upon retirement the individual can draw upon the money in the account.

Depending on how much is contributed and how well it is invested, the retiree could have more than enough for a comfortable retirement, or less than enough.

Assume you enter the labor force at the age of 35. Your job pays a fixed 50,000 per year for the next thirty years, after which you retire at age 65. This job provides no pension instead its your responsibility to ensure you save enough to maintain a dignified standard of living during your retirement. Finally also ignore taxes (both income tax and any possible deductions earned from participating in a pension plan) and we assume you die at age 95.

\section{Probability models of mortality}
As described above the  the key metric in a retirement scheme is how long people live, to ensure the pension plan is fully funded. If we are willing to stretch our idea of "experiment" the length of life may be considered a variable experiment result since people living under similar conditions will die at different unpredictable ages.

% notes
You probably see life insurance as a way to avoid having to worry about what will happen to your loved ones if you die. The insurance companies see it as a bet on how long you will live. As a business, they have to bet in such a way that their expectation is sufficiently positive for them to make an adequate profit.

Since the only framework available for discussing probabilities was games of chance, Christiaan Huygens conceived a life table as a lottery having 100 tickets of different values corresponding to the table’s entries. He then proceeded to calculate life expectancies using the rule for computing expectation that he had given in his earlier paper. For example, he stated that the number of chances that a person aged sixteen will die before age thirty-six equals 24, and that the number of chances that he will die after age thirty-six equals 16. Thus, in a fair game, you should bet 16 to 24, that is, 2 to 3, that the sixteen-year-old will die before age thirty-six.

The oldest example of this form of gambling is the life annuity, which dates back at least to the Roman Empire. The person who takes out a typical annuity is not really gambling: you simply pay a fixed amount by a specified date in order to receive regular payments for the remainder of your life. But to the authorities or, these days, the companies that sell you the annuity, it is very much a gamble: they are betting on your life. If you die before they have paid out to you the full value of your purchase price, they have made a profit on you; if you live a long time, they take a loss. Clearly, their task is to set the purchase price so that, when averaged over all their customers, they make an acceptable profit while avoiding charges of profiteering on people’s misfortunes.