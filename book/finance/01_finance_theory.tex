\chapter{Finance theory}
% TODO Missing finance concepts
%* simple imterest
% - present value under simple interst
% - interst paid at different time periods
%* compund interst
% - future value of compund intersts
% - calculation of the interst rate
% - present value under compund interst
%* interst rate
% - nominal and effective interst rate and converting between them
%* Annuities
% - future and present value
% - periodic payment and future value of annuity

\section{The time value of money}
We all know that money deposited into a savings account will earn interest.
Because of this ability we prefer to receive money today rather than the same 
amount in the future (since if we recived it today it could earn interest and therefor be worth more than the same amount recived in the future). This simple 
fact about money is the corner stone of all mthemtical finance theory and is 
known as the "time value of money". For example, assuming a $5\%$ interest rate, 
$100$ invested today will be worth $105$ in one year ($100$ multiplied by $1.05$).
Conversely, $100$ received one year from now is only worth $95.24$ today ($100$ 
divided by $1.05$).

\subsection{Interests}
When money is lent, borrowed or invested interest ($I$) is paid. The interest
paid is determined by three factors

\begin{enumerate}
    \item Amount borrowed (known as the principle or capital ammont)
    \item Length of time for which money is borrowed
    \item The rate at which interest is charged
\end{enumerate}

Interest is calculated by dividing the total time into periods (days, weeks,
months, quarters, half- years, years etc). For each period the money earns
interest. At the end of these periods the money is worth the original (P)
amount plus interest (I). This new amount will be referred to as the Future
Value (FV). The number of periods will be referred to as (N).

Interest can be simple by letting the principal stay the same after each
interest peiod
\[
    SimpleInterest(P,j,t) = P*(1+j*t)
\]
or it can be compund by adding the accrued interest to the principal e.g.
\[
    CompundInterest(P,j,m,t) = P*(1 + j/m)^(t*m)
\]

The present value, is a future amount of money that has been discounted to
reflect its current value, as if it existed today.
\[
    PresentValue(S,j,m,t) = S/((1 + j/m)^(t*m))
\]