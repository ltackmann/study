\chapter{Abstract algebra}
% TODO missing abstract algebra
% - go through first math chapter of "Symbolic C++" (commutative group, distributive, ..)
% - explain why these defenitions are needed

\section{Groups}


\section{Rings and Fields}
A ring os a ordered triple $(R, +, \cdot)$ consisting of a set R with two
binary operations $+$ (addition) and $\cdot$ (multiplication), satiesfying the
folloiwn conditions

\begin{enumerate}
\item the pair $(R, +)$ is a commutative group
\item multiplication is associative
\item it admits an identity (or unit) element, denoted by $I$
\item multiplication is distributive on both sides over addition, i.e.
\[
x \cdot (y + z) = x \cdot y + x \cdot z
\textnormal{ and }
(x+y) \cdot z = x \cdot z + y \cdot z
\textnormal{ for all } x,y,z \in R
\]
\end{enumerate}

\noindent A subring of $R$ is a subset $S$ of $R$ satiesfying:
\begin{enumerate}
\item $S$ is a subgroup of the additive group $R$
\item $x \in $ and $y \in S$ together imply $x \cdot y \in S$
\item $I \in S$
\end{enumerate}
Thus a subring is a ring

If an element $a \in R$ posses an inverse element with respect to
multiplication, i.e. if there exists a (unique) $a^{-1} \in R$ such that
\[
a \cdot a^{-1} = a^{-1} \cdot a = I
\]
then we say that a is an \textit{invertible element} of R.

The set of invertible elements of a ring $R$ is denoted by $R^*$. If every
non-zero element of $R$ is invertible, rhen $R$ is sait to be a
\textit{division ring}. A commutative division ring is called a\textit{field}.
$S$ is a \textit{subfield} of $F$ is $S$ is a subring of $F$ and
$x \in S, x \neq 0$ together imply $x^-1 \in S$.

Examples of rings are $\mathbb{Z}, +, \cdot$ and $\mathbb{Q}, +, \cdot$. Where
$\mathbb{Z}$ is subring of $\mathbb{Q}$. The only invertible elements of
$\mathbb{Z}$ is $1$ and $-1$ whereas all non-zero elements of $\mathbb{Q}$ are
invertible.

\subsection{Polynomial ring}
