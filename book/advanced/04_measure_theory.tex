\chapter{Measure theory}
% TODO missing measure theory concepts
% - Lebesgue measure is the standard way of assigning a measure to subsets of n-dimensional Euclidean space. For n = 1, 2, or 3, it coincides with the standard measure of length, area, or volume
% http://terrytao.files.wordpress.com/2012/12/gsm-126-tao5-measure-book.pdf

One of the most fundamental concepts in Euclidean geometry is measuring 
geometric figures in one or more dimensions. In one, two, and three dimensions, 
we refer to their measure as the length, area, or volume of respectively.

With the advent of analytic geometry, however, Euclidean geometry became 
reinterpreted as the study of Cartesian products $\mathrm{R}^d$ of the real line 
$\mathrm{R}$. Using this analytic foundation rather than the classical geometrical 
one, it was no longer intuitively obvious how to define the measure $m(E)$ of a 
generall subset $E$ of $\mathrm{R}^d$;

