\chapter{Arithmetic}\label{arit}
% TODO arithmetic missing concepts
% - Ratio (3 to 1 = 3:1 means that for each time there is 3 of one type there is 1 of another).
% -- if a and b are in ratio 5:6 means for each time there is 5 of a there is 6 of b.
% -- if asked how many of type a there are if there are 12 of b we can calculate as (5/6) * 2 = 10/12
% - inequalities http://www.mathsisfun.com/algebra/inequality.html
% - absolute value
% Commutative property of multiplication (changing the order of factors in a multiplication problem and see how it affects the product.)
% Distributive property of multiplication (decomposing the factors in multiplication problems and see how it affects the product.) (9 x 8 = 4x8 + 5x8 = 32 + 40 = 72, 12 * 5 = 6*5 + 6*5 = 30 + 30 = 60)
% Associative property of multiplication (changing the grouping of factors in multiplication problems and see how it affects the product. 2×3×7 = (2×3)×7)
% http://www.theguardian.com/science/alexs-adventures-in-numberland/2015/may/21/how-to-solve-the-maths-puzzle-for-vietnamese-eight-year-olds-that-stumped-parents-and-teachers
% - rates
% \glsentry{cube}{the cube of a number is the number multiplied by itself twice, eg 3 cubed = $3^3 = 3 × 3 × 3 = 27$.}
% \glsentry{integer}{Positive and negative whole numbers such as -34, -5, 0, 1, 17, 1021}
% \glsentry{product}{The result of multiplying together two or more numbers or things, eg the product of 4 and 5 is 20.}
%\glsentry{quotient}{The result of dividing one number by another. The quotient of 14 and 7 is 2 as 14/7 = 2.}

% primes and prime factorization (composite numbers)

% http://www.cs.bu.edu/fac/byers/courses/330/S13/handouts/05multiply.ppt
% http://youtu.be/w4Dy7v-Y5-Y (karetsuba muktiplication)
% https://en.m.wikipedia.org/wiki/Karatsuba_algorithm

% grade school multiplication algorithm 
% - http://www.cim.mcgill.ca/~langer/250/1-gradeschool.pdf
% - http://www.cim.mcgill.ca/~langer/250/2-binary.pdf
% http://jeremykun.com/2014/05/26/learning-to-love-complex-numbers/
% excercise: 2.6598 / 7.8 = 0,341
% 69\4 = 17.25
% TODO define remainder and quotient
% multi digit division and division with decimals

% exponents 
% - http://mathforum.org/library/drmath/view/55603.html 
% - http://primes.utm.edu/glossary/page.php?sort=BinaryExponentiation 

% adding and subtracting in scientific notatio (convert to same base first)

% https://www.khanacademy.org/computing/computer-science/cryptography/modarithmetic/a/the-quotient-remainder-theorem
% https://www.khanacademy.org/computing/computer-science/cryptography/modarithmetic/a/what-is-modular-arithmetic
% https://www.khanacademy.org/computing/computer-science/cryptography/modarithmetic/a/modular-inverses
% http://www.csee.umbc.edu/~stephens/203/PDF/3-4.pdf
% http://mathforum.org/library/drmath/view/52343.html

% shift left (multiply by base n times (in base two a single left shift is the same as multiplying by two)
% shift right (divide by base n times (in base two a single right shift is the same as dividing by two)

% GCD algorithms
% - binary https://en.wikipedia.org/wiki/Greatest_common_divisor#Binary_method
% - https://en.wikipedia.org/wiki/Lehmer%27s_GCD_algorithm
% - http://blog.janmr.com/2009/10/computing-the-greatest-common-divisor.html

% Decimal conversion
% - http://math.stackexchange.com/questions/46864/best-way-of-computing-the-decimal-representation-of-a-fraction-with-an-arbitrary
% - http://mathforum.org/library/drmath/view/63848.html
% - http://programmers.stackexchange.com/questions/192070/what-is-a-efficient-way-to-find-repeating-decimal

% Determine if a number have a finite precision. Repeating decimals in base ten are all fraction with the denominator having at least one prime factors other than two and five  a fraction will terminate if and only if the denominator has for prime divisors only 2 and 5 since every terminating decimal has the form n/10^e, for some e >= 0 as n/10^e = n/(2*5)^e = n/(2^e*5^e)

% Root algorithms
% http://www.mathpath.org/Algor/squareroot/algor.square.root.why.htm
% http://www.mathpath.org/Algor/algor.p-th.root.htm

% Long division algorithm
% http://www.mathpath.org/Algor/algor.long.div.htm
% - binary https://www.quora.com/How-do-I-divide-two-numbers-using-only-bit-operations-in-Java
% - http://mathlesstraveled.com/2008/09/07/rational-numbers-and-decimal-expansions/

% Integer square root
% - http://www.embedded.com/electronics-blogs/programmer-s-toolbox/4219659/Integer-Square-Roots
% - http://stackoverflow.com/a/30136377

% floor and ceiling http://blog.janmr.com/2009/09/useful-properties-of-the-floor-and-ceil-functions.html
% http://math.stackexchange.com/a/344818
% http://math.stackexchange.com/questions/1202456/how-to-calculate-the-integer-portion-of-a-fraction-using-only-div-and
% http://math.stackexchange.com/a/582983
% truncate (same as integer part http://planetmath.org/integerpart)  For positive numbers, trunc and floor give the same result. But for negative numbers, floor rounds down and trunc rounds up. This is because trunc always rounds toward 0.
% IntegerPart[x]+FractionalPart[x] is always exactly x.

% Fibonachi
% - http://blog.janmr.com/2011/03/evaluation-of-fibonacci-numbers.html

% Powers
% - http://blog.janmr.com/2011/01/evaluation-of-powers.html

% arithmetic 
% - http://blog.janmr.com/2011/10/multiple-precision-addition.html
% - http://blog.janmr.com/2011/10/multiple-precision-subtraction.html
% - http://blog.janmr.com/2011/11/basic-multiple-precision-multiplication.html
% - http://blog.janmr.com/2012/11/basic-multiple-precision-short-division.html




Numbers are an integral part of nature; a day can be split into $24$ equal parts, each of our hands has five fingers and so on. Thus even without any formal study of arithmetic humans have a basic understanding of numbers

\myindent The earliest known formal concepts of numbers dates back to the Babylonian, Egyptian and later Greek and Roman civilizations. Here tax collectors used tables to record tax payments and in the process came up with various standardised shorthands to record their work. For example if a citizen payed five animals a tax collector may have written \rom{1}\rom{1}\rom{1}\rom{1}\rom{1}. Each \rom{1} representing one animal. If later additional five animals where collected he may have recorded \rom{10} instead of \rom{1}\rom{1}\rom{1}\rom{1}\rom{1}\rom{1}\rom{1}\rom{1}\rom{1}\rom{1} as its easer to read. From these early shorthands rose the first numerical systems where it was agreed to let specific symbols represent a quantity. By far the most successful of these was the, still familiar, roman symbols \rom{1}, \rom{2}, \rom{3}, \rom{4}, \rom{5}, \rom{6}, \rom{7}, \rom{8}, \rom{9} and \rom{10}. 

\myindent Once number systems where established their properties was explored. The \rom{7} book of Euclid's elements published around 300 BC. deals extensively with elementary number theory including divisibility and the existence of \index{prime}{prime numbers}. The concept of rational (fractional) numbers was introduced early as a useful device for division, such as dividing the land belonging to a village among its inhabitants for harvesting. Egyptians used the word "nfr" to denote zero balance in accounting and later during the $7th$ century AD, negative numbers were used in India to represent debts.

\myindent The use of numbers in trading and science quickly outgrew the capabilities of early numerical systems. The romans for example had no concept of position where the same symbol is reused in different positions to denote different values. Thus they continuously had to invent new symbols for new numbers such as \rom{100} for $100$ and \rom{1000} for $1000$. As the system grew more complex it resulted in cryptic quantities such as \rom{11111} for $11111$. These shortcomings eventually led to the creation of simpler and better systems. Around $400 AD$ a new positional system began surfacing in India based on the now familiar ten numbers $0-9$. 

\section{Hindu numeral system}
The Hindu numeral system was invented between the $1st$ and $4th$ century and then later adopted by Persian and Arab mathematicians in the $9th$ century. From the $14th$ century Roman numerals gradually began to be replaced by it. The main advantage of the Hindu system is that it greatly simplifies operations such as addition and subtraction. Also its use of position to denote order of magnitude ensures it requires far fewer different symbols to represents numbers. For example the number $2506$ really represents
\[
2506 = 2 \times 1000 + 5 \times 100 + 0 \times 10 + 6 \times 1
\]
and the number $203$ is represented as 
\[
213 = 2 \times 100 + 1 \times 10 + 3 \times 1
\]
to add these we simply add the numbers in each position
\begin{align*}
2506 + 213 =& 2 \times 1000 + (5+2) \times 100 + (0+1) \times 10 + (6+3) \times 1 \\
           =& 2 \times 1000 + 7 \times 100 + 1 \times 10 + 9 \times 1 \\
           =& 2719 
\end{align*}
or more familiarly 
\begin{figure}[H]
\centering
\opadd{2506}{213}
\end{figure}

Using the exponent notation 
\[
10^{n} = \underbrace{10 \times \dots \times 10}_{n}
\]
that is $10^1 = 10$, $10^2 = 10 \times 10 = 100$ and $10^3 = 10 \times 10 \times 10 = 1000$ and taking advantage of that $10^0 = 1$ we can rewrite $2506$ as
\[
2506 = 2 \times 10^{3} + 5 \times 10^{2} + 0 \times 10^{1} + 6 \times 10^{0}.
\]
and in general we can write any possible integer $a_{n}a_{n-1} \cdots a_{1}a_{0}$ as
\begin{equation}\label{arit:int-system}
a_{n}a_{n-1} \cdots a_{1}a_{0} = a_{n} \times 10^{n} + a_{n-1} \times 10^{n-1} + \cdots + a_{1} \times 10^{1} + a_{0} \times 10^{0} 
\end{equation}
where each digit $a$ is a number between $0$ and $9$ and a sequence like $a_{3}a_{2}a_{1}a_{0}$ represents a four digit number such as $2506$. Using the \index{Sigma notation}{sigma notation} as a shorthand for sums 
\[
\sum_{k=1}^{n} k = \underbrace{1 + \dots + n}_{n}
\]
that is the sum $1 + 2 + 3 + 4 + 5 + 6$ can be written as 
\[
\sum_{n=1}^{6} n 
\]
and the sum $10^0 + 10^1 + 10^2 + 10^3$ can be written as
\[
\sum_{n=0}^{3} 10^n
\]
we can simplify \ref{arit:int-system} to
\begin{equation}
a_{n}a_{n-1} \cdots a_{1}a_{0} = \sum_{k=0}^{n} a_{k}10^{k}
\end{equation}

\section{Fractions and decimal numbers}
% TODO
% [Simplifying Fractions] (TODO examples 78/52
% [Converting fractions to decimals]

The concept of fractions predates the hindu numerals as way of dividing a whole into parts. For example a farmer may decide to split his land into $4$ parts storing the crops from the three parts and selling the crops from the remaining one part. Thus the fraction $\frac{3}{4}$ represents the crops to keep and the remaining $\frac{1}{4}$ those to sell and $\frac{4}{4}$ represents the entire land. In the fraction $\frac{3}{4}$ the number $3$ is known as the \index{Numerator}{numerator} and tells us that the fraction represents $3$ equal parts. The \index{Denominator}{denominator} $4$ tells us that $4$ parts make up a whole. Equally the fraction $5/4$ tells us that $4/4$ makes up a whole and we have an extra $1/4$ part left. As fractions represents equal parts we need to take care when performing arithmetic on them. Consider the meaning of splitting two equal size pizzas such that one is cut into $3$ parts and the other into $4$ parts. Clearly $3/3 = 4/4$ since getting all of the slices of either pizza would equal to getting the entire pizza. However one slice of the pizza cut into three pieces is greater than one from the pizza cut into four, so that $1/3 > 1/4$. Thus when comparing the size of fractions we need to take into account how many equal parts makes up whole and how many of those we have. In general we have
\begin{equation}
\frac{a}{b} < \frac{c}{d} \text{if and only if $ad < bc$}.
\end{equation}
so when comparing $3/4$ to $5/7$ we compute $3 \times 7 = 21$ and $5 \times 4 = 20$ and as $21 > 20$ we know that $3/4 > 5/7$. From this we also see that two fractions have the same value when they represent the same parts of a hole e.g. 
\begin{equation}
\frac{a}{b} = \frac{c}{d} \text{if and only if $a \times d = b \times c$}.
\end{equation}
Its interesting to note that a fraction preservers its value when we multiply the numerator and denominator with the same value that is 
\[
\frac{1}{2} = \frac{2 \times 1}{2 \times 2} = \frac{2}{4} 
\]
the rational from this being that getting one part of a pizza cut into halves results in the same amount of pizza as if receiving two slices of the same pizza cut into four parts. When adding or subtracting fractions we need to be extra vigilant. Adding $1/3$ to $1/4$ does not yield $2/7$. Instead we can only add or subtract two fractions when they represents the same division of a whole, that is they have a common denominator.
In this case we can multiply the numerator and denominator of $1/4$ with $3$ to obtain the common denominator $12$ observing that this new fraction has the same value as the old as 
\[
\frac{1}{4} = \frac{3 \times 1}{3 \times 4} = \frac{3}{12}
\]
Seminally we can multiply the numerator and denominator of $1/3$ with $4$ to obtain the same denominator i.e 
\[
\frac{1}{4} = \frac{4 \times 1}{4 \times 3} = \frac{4}{12}
\] 
and now we can add the two numbers as 
\[
\frac{1}{3} + \frac{1}{4} = \frac{4}{12} + \frac{3}{12} = 7/12
\] 
In general when adding fractions
\begin{equation}
\frac{a}{b} + \frac{c}{d} = \frac{a \times d + b \times b}{b \times d}
\end{equation}
and when subtracting them 
\begin{equation}
\frac{a}{b} - \frac{c}{d} = \frac{a \times d - c \times b}{b \times d}
\end{equation}
Here we make good use of a simple way of obtaining a common denominator by multiplying each fraction by the denominator of the other. 

% TODO describe decimals here
Place the decimal in the numerator and one in the denominator and multiply both with multiples of ten until the decimal point disapears e.g. $0.024 = 0.024/1 = 0.24/10 = 2.4/100 = 24/1000$

\begin{description}
\item [Multiplying fractions] $\frac{a}{b} * \frac{c}{d} = \frac{a*b}{c*d}$
\item [Dividing fractions] Division is equivalent to multiplying by the
reciprocal of the divisor fraction: $\frac{a}{b}/\frac{c}{d} =
\frac{a}{b} \cdot \frac{d}{c}$
\end{description}

\subsection{Mixed numbers}
Improper fractions such as $5/4$, where the numerator is larger than the denominator, are commonly represented as a combination of a whole number and a proper fraction called a mixed number e.g. $5/4 = 1\frac{1}{4}$.
\begin{description}
\item [Converting fractions to mixed numbers] Convert the numerator to a sum where the first part can be divided by the denominator and the last part has a lower value than the denominator e.g. $\frac{64}{5} = \frac{60 + 4}{5} = \frac{60}{5} + \frac{4}{5} = 12 \frac{4}{5}$
\item [Converting mixed numbers to fractions] Multiply the leading integer with the denominator and add it to the numerator e.g. $12 \frac{2}{3} = \frac{3 * 12 + 2}{3} = \frac{38}{3}$
\item [Adding and subtracting mixed numbers] A mixed number $a \frac{b}{c}$ is a short hand for $a + \frac{b}{c}$ so to add or subtract mixed numbers perform the operation step wise on the integer and fractional part of the numbers e.g. $6\frac{6}{12} - 3\frac{3}{5} = 6 + \frac{6}{12} - 3 - \frac{3}{5} = 2\frac{9}{10}$.
\item [Multiplying mixed numbers] Convert the mixed number to a fraction and multiply these e.g. $1\frac{1}{4} * 4 = \frac{20}{4} = 5$
\end{description}


The system is easily extended to handle decimal numbers such as $2506.12$. If we use the exponent notation
\[
10^{-n} = \frac{1}{10^{n}}
\]
that is $10^{-1} = \frac{1}{10} = 0.1$ and $10^{-2} = \frac{1}{100} = 0.01$. We can rewrite $2506.12$ as 
\begin{align*}
2506.12 =& 2 \times 1000 + 5 \times 100 + 0 \times 10 + 6 \times 1 + 1 \times 0.1 + 2 \times 0.01 \\
        =& 2 \times 10^{3} + 5 \times 10^{2} + 0 \times 10^{1} + 6 \times 10^{0} + 1 \times 10^{-1} + 2 \times 10^{-2} 
\end{align*}
Generally we represent a number with $n$ integer digits $a_{n}a_{n-1} \cdots a_{1}a_{0}$ and $m$ decimal digits 
$b_{1}b_{2} \cdots b_{m-1}b_{m}$ as 
\begin{equation}\label{arit:decimal-system}
a_{n}a_{n-1} \cdots a_{1}a_{0}.b_{1}b_{2} \cdots b_{m-1}b_{m} = \sum_{k=0}^{n} a_{k}10^{k}.\sum_{k=1}^{m} b_{k}10^{-k}
\end{equation}
From the above formula we also see why when a decimal number is multiplied by $10$, every figure moves one place to the left
\begin{align*}
1.25 \times 10 =& (1 \times 10^0 + 2 \times 10^{-1} + 5 \times 10^{-2}) \times 10 \\
               =& 1 \times 10^1 + 2 \times 10^0 + 5 \times 10^{-1} \\
               =& 12.5 
\end{align*}
and when divided by $10$, every figure moves one place to the right 
\begin{align*}
\frac{1.25}{10} =& 1.25 \times 10^{-1} \\
                =& (1 \times 10^0 + 2 \times 10^{-1} + 5 \times 10^{-2}) \times 10^{-1} \\
                =& 1 \times 10^{-1} + 2 \times 10^{-2} + 5 \times 10^{-3} \\
                =& 0.125 
\end{align*}

\section{Basic properties of numbers}
One of the many advantages of the positional number system is how it makes comparison of numerical values easy. In the roman numeral system this was only possible by memorising the quantities represented by each symbol; one simply had to remember that \rom{1000} was larger than \rom{100}. In the positional system we only need to compare the value of each digit in each position. To make this easy names for each position has been invented, some of these are listed below
\begin{figure}[H]
\centering
\begin{tabular}{|l|l|l|l|}
\hline
\textbf{Decimal name} & \textbf{Decimal position}  \\ \hline
thousands	         & $10^3 = 1000$               \\ \hline
hundreds	             & $10^2 = 100$                \\ \hline
tens	                 & $10^1 = 10$                 \\ \hline
ones	                 & $10^0 = 1$                  \\ \hline
tenths	             & $10^{-1} = 0.1$             \\ \hline
hundredths	        & $10^{-2} = 0.01$             \\ \hline
thousandths	        & $10^{-3} = 0.001$           \\ \hline
\end{tabular}
\end{figure}
To compare $1000$ to $999$ we simply begin from the highest digit and work our way down. Since the first number has $1$ in the thousands place and the second has a zero ($999= 0999$) we know that $1000$ is greater than $999$ and we write $1000>999$. Generally if one number $a$ is larger than another $b$ we write $a > b$ and if smaller we write $b < a$. If all the digits are the same then the numbers are equal and we write $a=b$. 

\myindent Sometimes, for example when comparing the speed of cars, we are only interested in how far one number is from another. If the speed limit is $100$ and you are going $70$ and another driver is going $130$, both of you will likely get a ticket. This is because the difference between you and everybody else is $30$. Not $-30$ and $30$ but $30$. This concept is called \index{absolute value}{absolute value}. The absolute value of a integer $x$ is written $\abs{x}$ and is simply the numerical value obtained by removing any negative sign. Formally we write
\begin{equation}
  \abs{x}=\begin{cases}
    x, & \text{if $x \geq 0$.}\\
   -x, & \text{if $x < 0$.}
  \end{cases}
\end{equation}
So the absolute value of $\abs{-30} = 30$ and that of $\abs{30} = 30$. 

\myindent In many cases its only necessary to compare numbers to a certain precision for example when comparing the population of one country to another we may be satisfied to count their populations in millions. This is the process of \index{Rounding}{rounding}. The idea is to replace a number by another approximately equal value that has a shorter or simpler representation. The procedure is as follows: \textit{Pick the number to round to as the last digit to keep, leave it the same if the next digit is less than $5$ or increase it by $1$ if the next digit is $5$ or more. Finally turn all following digets into zeros}. The basic algorithm is as follows
\begin{algorithm}[H]
    \caption{Rounding algorithm}
    \label{arit:rounding}
    \begin{algorithmic}[1]
        \Procedure{Round}{$n,i$} \Comment{Round $n$ to $i$'th place}
            \State $digits \gets digitsOf(n)$ \Comment{Array with digits of $n$}
            \State $inc \gets 0$
            \If{$digits[i-1] > 4$}           \Comment{Increase digit if above $4$}
                 \State $inc \gets 10^{i}$  
            \EndIf
            \While{$i > 0$}                    \Comment{Zero out remaining digits}
               \State $digits[i-1] = 0$      
               \State $i \gets i - 1$
            \EndWhile
            \State $r \gets numberOf(digits) $ \Comment{Turn array into number}
            \State $r \gets r + inc$     
            \State \textbf{return} $r$  \Comment{The rounded value of $n$}
        \EndProcedure
    \end{algorithmic}
\end{algorithm}
If we want to round the number $1997$ to the nearest ten ($i$ becomes $1$ as $10^1 = 10$) we first turn it into an array $\left[1,9,9,7\right]$ then examine the digit in the $i-1$ position which is $7$. As this is above $4$ we determine to increment the number by $10$. The lower digits are then zeroed out to $\left[1,9,9,0\right]$ and turned into the number $1990$ which we add the $10$ to form the final result $2000$. Besides the standard rounding method mentioned above two other popular methods exists called \index{Floor}{floor} and \index{Ceiling}{ceiling}. These maps a decimal number to the largest previous or the smallest following integer, respectively. More precisely, $floor(x) = \lfloor x \rfloor$ is the largest integer not greater than $x$ and $ceiling(x) =  \lceil x \rceil$ is the smallest integer not less than $x$. The following table should clarify how their difference
\begin{figure}[H]
\centering
\begin{tabular}{|l|l|l|l|}
\hline
\textbf{Value}  & \textbf{Floor} & \textbf{Ceiling} & \textbf{Fractional part} \\ \hline
2.4	            & 2	            & 3	              & 0.4                      \\ \hline
2.9	            & 2	            & 3	              & 0.9                      \\ \hline
-2.7	            & -3	            & -2	              & 0.3                      \\ \hline
-2	            & -2	            & -2	              & 0                        \\ \hline
\end{tabular}
\end{figure}

Another device for easily comparing and writing numbers was invented when scientist began studying very large and small quantities. Using ordinary decimal numbers became inconvenient. To remedy this \index{Scientific notation}{scientific notation} was invented in which all numbers are written in the form
\[
a \times 10^b
\]
where the exponent $b$ is chosen so its absolute value is at least one but less than ten ($1 \leq |a| < 10$). Thus $350$ is written as $3.5×10^2$. Sometimes the so called \textit{E notation} is used where we write $e$ in the exponent instead of $10$ that is
\[
3.0 \times 10^{-9} = 3.0e−9
\]
this notation was originally invented as calculators could not display exponents correctly. Note that the $E$ or $e$ should not be confused with Euler's number $e$ which holds a completely different significance. The following table are examples of scientific notation:
\begin{figure}[H]
\centering
\begin{tabular}{|l|l|l|l|}
\hline
\textbf{Decimal notation}   & \textbf{Scientific notation} \\ \hline
2                           & $2 \times 10^0$              \\ \hline
300                         & $3 \times 10^2$              \\ \hline
4,321.768                   & $4,321768 \times 10^3$       \\ \hline
−53,000                     & $−5,3 \times 10^4$           \\ \hline
6,720.000.000	           & $6,72 \times 10^9$           \\ \hline
0.2	                        & $2 \times 10^{−1}$           \\ \hline
0,000.000.007.51	           & $7,51 \times 10^{−9}$        \\ \hline
\end{tabular}
\end{figure}
This form allows easy comparison of numbers, as the exponent $b$ gives the number's \index{order of magnitude}{order of magnitude}. For example, the order of magnitude of $1500$ is $3$, since $1500$ is written as $1.5 × 10^3$. When adding and subtracting two numbers in scientific notation, we must adjust the two values so that their exponents are the same. For example, when adding $1.23e1$ and $4.56e0$, we first adjust $4.56e0$ to $0.456e1$ and then add it to $1.23e1$ which produces $1.686e1$.

\section{Arithmetic in the Hindu numeral system}
As mentioned the perhaps biggest advantage of the Hindu numeral system was the ease with which merchants and scientist could carry out calculations such as addition, subtraction, multiplication and division

The order of arithmetic operators is as follows
\begin{enumerate}
\item Parentheses and exponents
\item Multiplication and division
\item Addition and subtraction
\end{enumerate}
When multiple operations exists on the same level you do the one that it closest on the left, for example
\[
2-10+8-(-2)+(-10) = 2 - 10 + 8 + 2 - 10 
= (2 - 10 + 8) + 2 - 10 = (-8 + 8) + 2 - 10 = 0 + 2 - 10 = -8
\]


\subsection{Addition}
% TODO missing addition concepts
% - describe adding decimals
Addition represents the operation of adding objects to a collection. For example, $2+3$ can be though of as if you have $2$ apples and someone give you $3$ more, then you have $5$ apples in total. Assuming $b$ is positive then on a number line $a+b$ represents starting on number $a$ and moving $b$ places to the right. Similar if $b$ is negative then $a+b$ represents starting on number $a$ and moving $b$ places to the left. Unlike subtraction the addition operations is both commutative (that is $a + b = b + a$) and associative (that is $a + (b + c) = (a + b) + c$)

\begin{description}
\item [Adding integers] We add integers by adding the ones, tens,hundreds etc. in each number individually, starting from the right. If the result of addition is above $10$ then we subtract $10$ from the result and add one (known as a carry) to the next group to be added. For example
\begin{figure}[H]
\centering
\opadd{143}{89}
\end{figure}
That is we add $3+9=12$ as the result is above $10$ we subtract $10$ from it and add one tenth to the tenths place. In the tenth place we now add $1+3+8=12$ which again is above $10$, so we subtract $10$ from it and add $1$ to the hundreds place (ten tens being one hundred) and so finally we add $1+1$ in the hundreds place.

To understand why this procedure works consider that the number $143 = 100 + 40 + 3$ and $89 = 80 + 9$ so we can do
\[
\begin{align}
143 + 89 &= (100 + 40 + 3) + (80 + 9) \\
         &= 100 + (40+80) + (3+9)     \\
         &= 100 + 120 + 12            \\
         &= 232
\end{align}
\]
\item [Adding decimals] We add decimals similar to integers by starting
from the rightmost decimal.
\begin{figure}[H]
\centering
\opadd{9.087}{15.924}
\end{figure}
\end{description}

\begin{algorithm}[H]
    \caption{Add two N digit numbers a and b represented as arrays of digits}
    \label{arit:addition}
    \begin{algorithmic}[1]
        \Procedure{Addition}{$a,a$} 
            \State $carry \gets 0$ 
            \For{$i=0 \text{ to } N-1$} 
                \State $sum \gets a[i] + b[i] + carry$
                \State $carry \gets sum / 10$ \Comment{TODO}
                \State $r[i] \gets sum \% 10$ \Comment{TODO}
            \EndFor
            \State $r[N] \gets carry$
            \Return $r$  
        \EndProcedure
    \end{algorithmic}
\end{algorithm}
% TODO explain div and mod                 \State $carry \gets (a[i] + b[i] + carry)/10$
% r[i] \gets (a[i] + b[i] + carry) % 10

\subsection{Subtraction}
Subtraction represents the operation of removing objects from a collection. For example, $5-2$ can be though of as if you have $5$ apples and take $2$ away, then you have $3$ apples left. Assuming $b$ is positive then on a number line $a-b$ represents starting on number $a$ and moving $b$ places to the left. Similar if $b$ is negative then $a-b$ represents starting on number $a$ and moving $b$ places to the right.

Subtraction is not commutative, that is $a - b \neq b - a$ instead we can reverse the subtraction order by $a - b = -(b - a) = -b + a$. Further subtraction is not associative so $a  - (b - c) \neq (a - b) - c$ instead it holds $a - b - c = a - (b + c)$ and $a - b - c - d = a - (b + c + d)$ and so on.
\begin{description}
\item [Subtracting integers] We subtract numbers by subtracting each
ones, tens, hundreds etc. individually, starting from the right, if the
subtraction yields a result less than zero then we borrow ten from the
number to the left by decreasing its value by one:
\begin{figure}[H]
\centering
\opsub{133}{89}
\end{figure}
\item [Subtracting decimals]
% TODO subtracting decimals
\end{description}

\subsection{Multiplication}
Multiplication of two whole numbers is equivalent to the addition of one of them with itself as many times as the value of the other one; for example, $3 \cdot 4$ can be calculated by adding $4$ to itself $3$ times:
\[
3 \cdot 4 = 4 + 4 + 4 = 12
\]
it can also be calculated by adding $4$ copies of $3$ together:
\[
3 \cdot 4 = 3 + 3 + 3 + 3 = 12
\]
Formally we say that multiplication is both commutative (i.e. the order in which two numbers are multiplied does not matter $a \cdot b = b \cdot a$) and associative $a \cdot (b \cdot c) = (a \cdot b) \cdot c$. Multiplication has other important properties:
\begin{description}
\item [Distributive property] Holds with respect to multiplication over addition. This identity is of prime importance in simplifying algebraic expressions: $x\cdot(y + z) = x\cdot y + x\cdot z$
\item [Identity element] The multiplicative identity is $1$; anything multiplied by one is itself. This is known as the identity property: $x\cdot 1 = x$
\item [Zero element] Any number multiplied by zero is zero. This is known as the zero property of multiplication: $x\cdot  0 = 0$
\end{description}

% TODO missing multiplication areas
% - factors and products
% - explain operations of single and multi diget
% - explain meaning of multiplying negative numbers
% - explain techniques for muktiplication
\begin{description}
\item [Multiplying integers] We multiply integers by multiplying the ones, tens, hundreds etc. in each number individually, starting from the right. If the result of multiplication is above $10$ we subtract $10$ from the result and add one (known as a carry) to the next group to be multiplied. For example
\begin{figure}[H]
\centering
\opmul[displayintermediary=None]{142}{3}
\end{figure}

That is we first multiply $3 \cdot 2 = 6$ in the ones place, then in the tenth place we multiply $3 \cdot 4 = 12$ which is above $10$, so we subtract $10$ from it and add $1$ to the hundreds place and so finally we multiply $3 \cdot 1 = 3$ and add the carry of $1$ to it in order to get $4$ in the hundreds place.

To understand why this procedure works consider that the number $142 = 100 + 40 + 2$ so really we have
\[
\begin{align}
143 \cdot 3 &= (100 + 40 + 2) \cdot 3  \\
            &= (100 \cdot 3) + (40 \cdot 3) + (2 \cdot 3)  \\
            &= (100 \cdot 3) + (10 \cdot 4 \cdot 3) + 6  \\
            &= (100 \cdot 3) + (10 \cdot 12) + 6  \\
            &= (100 \cdot 3) + (100 + 20) + 6  \\
            &= (100 \cdot 4) + 20 + 6  \\
            &= 426
\end{align}
\]

\item [Multiplying decimals] To multiply decimals convert them into integers, multiply them and then convert the result back to a decimal i.e.
\begin{itemize}
\item $8 * 0.8 = 8 * \frac{8}{10} = \frac{64}{10} = 6.4$
\item $2.91 * 3.2 = \frac{291}{100} * \frac{32}{10} = \frac{291 * 32}{1000} =
\frac{9312}{1000} = 9.312$
\end{itemize}
\end{description}

When multiplying numbers of different precision care should be taken when deciding on the preccion of the result. To facilitate this the concept of \index{significant figures}{significant figures} has been invented. The significant figures of a number are those digits that carry meaning contributing to its precision. The procedure for identifying significant figures is as follows 
\begin{enumerate}
\item Identify the non-zero digits and any zeros between them. These are all significant.
\item Leading zeros are not significant.
\item If its a decimal, then trailing zeros are significant.
\end{enumerate}
for example
\[
1.234 \times 2.0 = 2.468\cdots \equiv 2.5
\]
here the first factor $1.234$ has four significant figures and the second $2.0$ has two significant figures. The factor with the least number of significant figures is the second one with only two, so the final result should also have a total of two significant figures.




\subsection{Division}
% TODO missing division
% - using  prime factorisation to simplify numbers (see number theory)
% - factors
% - GCD and LCM
% - explain why division algo works
% - division commutativity and associative
\begin{description}
\item [Dividing improper fractions] Division is really counting the number of repeated subtractions of the divisor into the dividend e.g. $3024/42 = 72$ as shown below
\newline
\[
\begin{array}{*{6}{>{\hfill}m{7mm}}}
 &
     &
         0&
             0&
                 7&
                     2\\\cline{3-6}
4&
  2\big)&
        3&
             0&
                 2&
                     4\\\cline{1-2}
 &
     &
        2&
            9&
                4&
                     \\\cline{3-5}
 &
     &
         &
             &
                8&
                     4\\
 &
     &
         &
             &
                 8&
                     4\\\cline{5-6}
\end{array}
\]
\item [Dividing proper fractions] As proper fractions always represent values parts less than 1 whole we know the result must be a decimal beginning with 0.
\[
\begin{array}{*{7}{>{\hfill}m{7mm}}}
 &
     &
         &
           0,&
                7&
                    0&
                        3\\\cline{3-7}
2&
  7\big)&
        1&
           9,&
                0&
                    0&
                        0\\\cline{1-2}
 &
     &
        1&
            8&
                9&
                     &
                         \\\cline{3-5}
 &
     &
         &
             &
                1&
                    0&
                        \\
 &
     &
         &
             &
                 &
                     0&
                        \\\cline{5-7}
 &
     &
         &
             &
                 1&
                     0&
                         0\\
 &
     &
         &
             &
                 &
                     8&
                         1\\\cline{6-7}
 &
     &
         &
             &
                 &
                     1&
                         9\\
\end{array}
\]
\item [Dividing decimals] To divide decimals first convert the denominator into a integer then divide the resulting fraction as shown above e.g. $3.3534/0.81 = 335.34/81$
\[
\begin{array}{*{7}{>{\hfill}m{7mm}}}
 &
     &
        0&
            0&
                4,&
                    1&
                        4\\\cline{3-7}
8&
  1\big)&
        3&
            3&
                5,&
                    3&
                        4\\\cline{1-2}
 &
     &
        3&
            2&
                4&
                     &
                         \\\cline{3-5}
 &
     &
         &
             1&
                1&
                    3&
                        \\
 &
     &
         &
             &
                 8&
                     1&
                        \\\cline{5-7}
 &
     &
         &
             &
                 3&
                     2&
                         4\\
 &
     &
         &
             &
                 3&
                     2&
                         4\\\cline{6-7}
\end{array}
\]

\end{description}

\subsection{Divisibility tests}
% TODO proove these
The following rules can test numbers for divisibility
\begin{description}
\item [Divisible by $2$] if the last didget is divisible by $2$.
\item [Divisible by $3$] if the sum of the digits is divisible by $3$.
\item [Divisible by $4$] if the number formed by the last two digits is
divisible by $4$.
\item [Divisible by $5$] if the last digit is either $0$ or $5$.
\item [Divisible by $6$] if divisible by $2$ and $3$.
\item [Divisible by $9$] if the sum of the digits is divisible by $9$.
\item [Divisible by $10$] if the last digit is $0$.
\end{description}

\subsection{Remainder}
The remainder from the division $a/b$ is represented mathematically as $a \textrm{mod} b$, for example $9/2 = 4$ and $9 mod 2 = 1$. In general to the find the result of $a \textrm{mod} b$ we follow these steps:
\begin{enumerate}
\item Construct a clock with size $b$
\item Start at 0 and move around the clock $a$ steps (If the number is positive we step clockwise, if it's negative we step counter-clockwise).
\item Wherever we land is our solution.
\end{enumerate}
For example $7 mod 2 = 1$ since we can make a clock with numbers $0,1$ then start at $0$ and go through $7$ numbers in a clockwise sequence $1,0,1,0,1,0,1$. Similar $-5 mod 3 = 1$ since we make a clock with numbers $0,1,2$ then start at $0$ and go through $5$ numbers in counter-clockwise sequence  $2,1,0,2,1$




this let to the general rule that
\[
x^m \times x^n = x^{m+n}
\]
Similarly, dividing exponents $\frac{x^3}{x^2} = \frac{x \times x \times x}{x \times x} = x = x^1 = x^{3-2}$ let to the generalisation that 
\[
\frac{x^m}{x^n} = x^{m-n}
\]
Early in the development of exponent notation the above rule encountered issues when the exponent of the denominator was greater than that of the numerator, as in $x^3/x^5$, where the rule would give us $x^{3-5} = x^{-2}$. However by letting $x^{-n} = 1/x^n$ we obtain $x^{3-5} = x^{-2} = 1/x^2$ which agrees with the results of dividing $x^3$ by $x^5$ directly. When $m = n$ the above rule leads $x^m/x^n = x^{m-n} = x^0$ which we must define as one. 

\section{Exponents and logarithms}\label{arit:exp}
If we multiply 3 by itself 4 times we get
\[
3 \cdot 3 \cdot 3 \cdot 3 = 81
\]
A more concise way of writing this is to say
\[
3^4 = 81
\]
where 3 is the base and the superscript 4 is called the power or exponent.

% TODO compare list of rules with that from math picture taken

\begin{description}
\item [Squares $b^2$] means $b \cdot b$ and is read \emph{b squared} because $b^2$ is the area of a square whose side has length $b$.
\item [Cubes $b^3$] means $b \cdot b \cdot b$ and is read \emph{b cube} because $b^3$ is the volume of a cube whose side has length $b$.
\item [Powers of $10$] If the power is positive the result is one followed by as many zeros as the number in the exponent $10^7 = 10000000$. If the exponent is negative we have as many zeroes as the exponent followed by one with the first zero being before the comma the rest after $10^+7 = 0.0000001$
\item [Negative powers] $a^{-n} = \frac{1}{a^n}$
\item [Products of powers] $a^n \cdot a^m = a^{n+m}$ this can be seen as $5^3 \cdot 5^2 = (5 \cdot 5 \cdot 5) \cdot (5 \cdot 5)= 5^5$
\item [Products of unlike powers] $a^n \cdot b^n = (a \cdot b)^n$ this can be seen as $(2 \cdot 2 \cdot 2) \cdot (3 \cdot 3 \cdot 3) = (2 \cdot 3) \cdot (2 \cdot 3) \cdot (2 \cdot 3) = (2 \cdot 3)^3$
\item [Fractions of powers] $\frac{a^n}{a^m} = a^{n-m}$ this can be seen as $\frac{5^3}{5^2} = \frac{5 \cdot 5 \cdot 5}{5 \cdot 5} = 5^1$
\item [Powers of fractions] $\left(\frac{a}{b}\right)^n = \frac{a^n}{b^n}$ this can be seen as $\left(\frac{2}{3}\right)^3 = \frac{2}{3} \cdot \frac{2}{3} \cdot \frac{2}{3} = \frac{2^3}{3^3}$
\item [Powers of powers] $(a^m)^n = a^{(m \cdot n)}$ this can be seen as $(5^2)^3 = 5^2 \cdot 5^2 \cdot 5^2 = 5^{6}$
\item [Fractional powers] $a^{\frac{x}{y}} = (a^{\frac{1}{y}})^x$. This can be related to roots since $\sqrt{a} \cdot \sqrt{a} = a$ can be satisfied as $a^{x} \cdot a^{x} = a^{x+x} = a^{1}$ only when $x = 1/2$. That is $\sqrt{a} = a^{1/2}$. Generally we have $\sqrt[n]{a} = a^{1/n}$.
\item [Simplifying radicals] Radicals such as $\sqrt{32}$ are simplified by rewriting them to lower terms using the above rules e.g. $3\sqrt{8} - 6\sqrt{32} = 3\sqrt{4}\sqrt{2} - 6\sqrt{16}\sqrt{2} = 6\sqrt{2} - 24\sqrt{2} = -18\sqrt{2}$. Often it can be constructive to rewrite radicals using their prime factorisation e.g. $\sqrt[3]{81} = \sqrt[3]{9 \cdot 9} = \sqrt[3]{3 \cdot 3 \cdot 3 \cdot 3} = 3\sqrt[3]{3}$
\item [Exponentiation of fractions] If $n$ is a non-negative integer, then
$\left(\frac{a}{b}\right)^{n} = \frac{a^n}{b^n}$ and if $a \neq 0$ then
$\left(\frac{a}{b}\right)^{-n} = \frac{1}{\left(\frac{a}{b}\right)^n} =
\frac{1}{\frac{a^n}{b^n}} =\frac{b^n}{a^n}$.
\end{description}

The use of exponents was greatly boosted by the Scottish landowner {John Napier}\index{John Napier} (1550-1617) who invented {logarithms}\index{logarithms} while searching for a way to simplify computation of large numbers. His idea was to utulize the $x^n \times x^m = x^{n+m}$ property of exponents to simplify multiplication and division as it was generally agreed that multiplication and devision was more difficult than addition and subtraction. 

His line of thought was this: If we could write any positive number as a power of some given, fixed number (later to be called a base), then multiplication and division of numbers would be equivalent to addition and subtraction of their exponents. Furthermore, raising a number to the nth power (that is, multiplying it by itself n times) would be equivalent to adding the exponent n times to itself—that is, to multiplying it by n—and finding the nth root of a number would be equivalent to n repeated subtractions—that is, to division by n. In short, each arithmetic operation would be reduced to the one below it in the hierarchy of operations, thereby greatly reducing the drudgery of numerical computations. Let us illustrate how this idea works by choosing as our base the number 2. Table 1.1 shows the successive powers of 2, beginning with n = -3 and ending with n = 12. Suppose we wish to multiply 32 by 128. We look in the table for the exponents corresponding to 32 and 128 and find them to be 5 and 7, respectively. Adding these exponents gives us 12. We now reverse the process, looking for the number whose corresponding exponent is 12; this number is 4,096, the desired answer. As a second example, supppose we want to find 45. We find the exponent corresponding to 4, namely 2, and this time multiply it by 5 to get 10. We then look for the number whose exponent is 10 and find it to be 1,024. And, indeed, 45 = (22)5 = 210 = 1,024.

% TODO describe logarithmic base
% TODO describe computation with logarithmeic tables and how they where constructed 
% log (ab) = log a + log b
% log (a/b) = log a - log b
%log a^n = n log a,

\subsection{Estimating roots}
\begin{description}
\item [Square roots]
% TODO long hand square root algorithms
% - http://xlinux.nist.gov/dads/HTML/squareRoot.html
% - http://mathforum.org/library/drmath/view/52610.html
% - http://www.embedded.com/electronics-blogs/programmer-s-toolbox/4219659/Integer-Square-Roots
\item [Cube roots] find the prime factors and see if any come three times. For example $\sqrt[3]{3430}$ is divisible by $10$ and there for have $5$ and $2$ as prime factors, $343 = 7^3$ so we have $\sqrt[3]{3430} = \sqrt[3]{2 \cdot 5 \cdot 7 \cdot 7 \cdot 7} = 7 \cdot \sqrt[3]{10}$
\end{description}



\section{Percents}
% TODO percents missing concepts
% - relationship between percents, fractions and divisions (100 * 1.05 = 105, 100/1.05 = ??
% - pie costs 11 price drops 45 how much do you save (11 * 0.45 = what you save, 11 * 0.55 = what it will cost)
% - wholesale price and markup (explain concepts)




\section{General Numeral systems}
% TODO move to later number section
% TODO binary numbers http://www.wildbunny.co.uk/blog/2012/11/07/understanding-binary/
As time passed positional numbers systems was generalized such that the digits ($0-9$ in the decimal system) could be changed to accomedate different purposes. For example electric current can be in only two states on (power is flowing) or off (no power). Thus in cumputers its convinient to represent numbers using only two digits $0$ and $1$. To represent higher numbers multiple wires in either on or off states can be used for example if we have $4$ wires and all are on we can represent
\[
  1 \times 2^0 + 1 \times 2^1 + 1 \times 2^2 + 1 \times 2^3 = 1 + 2 + 4 + 8 = 15
\]
If all wires are off we have
\[
  0 \times 2^0 + 0 \times 2^1 + 0 \times 2^2 + 0 \times 2^3 = 0 + 0 + 0 + 0 = 0
\]
if the first and last are on we have

Thus with a four digit number in a binary number system we can represent the integers $0-15$ or $2^4$ different values. 

The generalized binary number looks like this where each bit bits, short for binary digits, is either 0 or 1.

% TODO generalized base conversion
A simple way to convert decimal to binary is the even/odd -divide by two algorithm. This algorithm uses the following steps: If the number is even, emit a 0. If the number is odd, emit a 1. Divide the number by 2 and throw away any fractional component or remainder. If the quotient is 0, the algorithm is complete. If the quotient is not 0 and is odd, insert a 1 before the current string; if the number is even, prefix your binary string with 0. Go back to step 2 and repeat.

binary numbers are very verbose you need 8 bits to represent the number 202. Although we can convert between decimal and binary, the conversion is not a trivial task and since computers only understands binary at their corr, such a conversion would have to happen allot. The hexadecimal (base 16) numbering system solves many of the problems inherent in the binary system. Hexadecimal numbers offer the two features we're looking for: They're very compact, and it's simple to convert them to binary and vice versa. 

Each hexadecimal digit can represent one of 16 values between 0 and 1510. Because there are only 10 decimal digits, we need to invent 6 additional digits to represent the values in the range 1010..1510. Rather than create new symbols for these digits, we'll use the letters A..F. 

Ag first base 16 may seam hard to calculator with but remembering $(a^m)^n = a^(n \times m)$ gives us that $16^2 = (2^4)^2 = 2^8 = 256$

% Decimal Binary  Hexadecimal  
% 0       0000  $0  
% 1 0001  $1  
% 2 0010  $2  
% 3 0011  $3  
% 4 0100  $4  
% 5 0101  $5  
% 6 0110  $6  
% 7 0111  $7  
% 8 1000  $8  
% 9 1001  $9  
% 10 1010  $A  
% 11 1011  $B  
% 12 1100  $C  
% 13 1101  $D  
% 14 1110  $E  
% 15 1111  $F  

To convert a hexadecimal number into a binary number, simply substitute the corresponding 4 bits for each hexadecimal digit in the number. For example, to convert ABCD into a binary value, simply convert each hexadecimal digit according to Table 2-1, as shown here: A  B  C  D  Hexadecimal  1010  1011  1100  1101  Binary  To convert a binary number into hexadecimal format is almost as easy. The first step is to pad the binary number with zeros to make sure that there is a multiple of 4 bits in the number. For example, given the binary number $1011001010$, the first step would be to add 2 bits to the left of the number so that it contains 12 bits. The converted binary value is $001011001010$. The next step is to separate the binary value into groups of 4 bits, for example, $0010\_1100\_1010$. Finally, look up these binary values in the table above.

Note that shifting a value to the left is the same thing as multiplying it by its radix. For example, shifting a decimal number one position to the left multiplies it by $10$ (the radix). Because the radix of a binary number is $2$, shifting it left multiplies it by $2$. If you shift a binary value to the left twice, you multiply it by $2$ twice (that is, you multiply it by $4$). If you shift a binary value to the left three times, you multiply it by $8$ ($2 \times 2 \times 2$). In general, if you shift a value to the left $n$ times, you multiply that value by $2n$.

\myindent In the generalised positional number system we use the \index{base} (or \index{radix}) to represent the different symbols available to each digit within that system. For example, the decimal system has a radix of $10$, that is any diget in a number can have values $0-9$. In contrary computers work using electronic circuits and thus digets are represented with radix $2$. Other commonly used number systems are hexadecimal wich uses radix $16$ with digets represented with the numbers $0-9$ and letters $A-F$. Generally we have

\subsection{Decimal representation}
\[
(a_{n}a_{n-1} \cdots a_{1}a_{0} . c_{1}c_{2} \cdots c_{m-1}c_{m}) =
    \sum_{k=0}^{n}a_{k}b^{k} + \sum_{k=1}^{m}c_{k}b^{-k}
\]
The numbers $b^{k}$ and $b_{k}$ are the weights of the corresponding digits. The position $k$ is the logarithm of the corresponding weight $w$, that is $k = \log_{b} w = \log_{b} b^k$

% TODO repeating decimals 
% - 0.77777 = 7/9
% - 1.8911111 
% - x = 1.8911111 
% - 100x = 189.11111 
% - 100x-x = 189.11111 - 1.8911111 
% - 99x = 187.22
% - x = 187.22/99 = 189.11111 


For example to write $1/3$ in base $16$ we get
% TODO http://en.wikipedia.org/wiki/Hexadecimal#Conversion

\subsection{Conversion between systems}
% - Number bases (convert 1/2 from base 10 base 6)

\newpage
\section{Exercises}

% The sum of four consequtive odd integers is 136 what are the integers
% 136 = x + (x+2) + (x+4) + (x+6) + (x+8) where x=2n+1
% 4x + 12 = 135
% 4x = 124
% x=31
% so 31 + 33 + 35 + 37 
%% EXCERCISES
% [Addition]
% - Single, multi and decimal
% [Multiplication]
% - Single, multi and decimal
% [Subtraction]
% - Single, multi and decimal
% [Division]
% - Single, multi and decimal
% 144.95/65
% [Modular arithmetic]
% - Dorothy gives a piece of candy to each of her friends in the following order and then starts over from the beginning: the Scarecrow, the Tinman, the Cowardly Lion, and Toto. Who gets the twenty-fifth piece of candy? (the Scarecrow gets it as we can finish 6 cycles of handing out candy and then begin in the first person in the group again)
% - Vincent van Gogh is making one-color versions of all his paintings. He uses the colors in the following order and then starts over from the beginning: red, blue, pink, green, and purple. What color is Painting 31 ? (Answer: red)
% [Scientific notation]
% - Express this quotient in scientific notation: \frac{7.0×10^{−4}}{5.320×10^{−2}
% - The volume of the Earth is 1.1⋅10^21 cubic meters. The volume of the sun is 1.4⋅10^27 cubic meters. How many Earths could fit inside the sun?
% - A strainbif bqcferia triples every hour,mif fhere initially ia 200bqcteria how many are fhere afger four hours (ans after one hour there are 3*200 after two there are 3*3*200 after four ghere are 3^4*200)

% To put the number in scientific notation we need to rewrite it as a number between 1 and  10, multiplied by a power of 10 since 0.4920 is smaller than 1 so we need to change how this number is written for it to meet the definition of scientific notation.

Before we get to the integers though we’re going to start with the naturals. The naturals are the numbers 0, 1, 2, 3, … and so on; they’re the “counting numbers”. We can loosely[1. A proper definition would also include restrictions like “zero is not the successor of any number” and “unequal numbers have unequal successors” and so on, but we don’t need to take a formal axiomatic approach for the purposes of this series.] define the natural numbers recursively:

- 0 is a natural number.
- The successor of a natural number is a natural number.

There are lots of ways to represent natural numbers. A standard way to represent the naturals in set theory is to say that a natural number is represented by the set of all natural numbers smaller than it. So zero, having no natural numbers smaller than it, is the empty set, {}. One has only zero smaller than it, so it is {{}}. Two has zero and one smaller than it, so it is {{},{{}}}.  Three is {{}, {{}}, {{},{{}}}}. 

Ok, addition and multiplication are in the can. Those are the easy ones because the set of natural numbers is “closed” over those operations. That is, if you have any two naturals then both their sum and their product is also a natural. Not so subtraction! To see why, first we have to state what subtraction really is.

Subtraction relates three numbers: the minuend and the subtrahend, which are given, and the difference, which is the result. When we say m - s = d what we are really saying is that d is the solution to the equation m = s + d. But for, say, m as 2 and s as 3 there is no solution d which is still a natural. The integers are closed over subtraction, but the naturals are not.

The purpose of the division and remainder operators is to take two natural numbers, x and y, and find two natural numbers called the quotient and remainder such that:


The purpose of the division and remainder operators is to take two natural numbers, x and y, and find two natural numbers called the quotient and remainder such that x/y means 

x = y * q + r
r < y
Clearly if y is zero then there is no remainder smaller than it, so on those grounds alone we can disqualify division by zero. But even if we allowed a zero remainder in the case where y was zero, then we either have infinitely many solutions for q if x is zero, or no solutions for q if x is not zero. Thus division by zero is illegal.

The integers are an extension to the naturals that closes them over subtraction. Rather than do anything fancy, I’m simply going to say that an integer is a natural with an associated sign, and that zero is always “positive”. We don’t want to end up in a situation where there are two forms of zero, since that is confusing. 

the \% operator is the remainder operator. When extending the division and remainder operators from naturals to integers, I require that the following conditions be true for dividend x, non-zero divisor y, quotient q and remainder r:

x == y * q + r
if x and y are non-negative then 0 <= r < y
(-x) / y == x / (-y) == -(x / y)
Let’s take a look at the consequences of these three facts.

First off, when the sign of an operand changes then the quotient changes sign but not magnitude.

TODO check that solvr comply with this
11 / 7 == 1, so -11 / 7 == -1 and 11 / -7 == -1 and -11 / -7 == 1.

I hope this is reasonable; it would be very strange if -11 / 7 and -(11 / 7) had different values.

TODO add these words to subtraction in the arithmetic chapter

TODO understand this

“If you don’t like that — if you’d rather -11 % 7 be 3” should be “If you don’t like that — if you’d rather -11 % 7 be 4”
No, it’s not a typo. The point Eric is making is that remainder and modulus are not the same: -11 is congruent to 3 modulo 7, but the remainder is -4. The way to think of modular arithmetic is like a cycle: the congruence class for 3 is 3, 3+7 = 10, 3+7+7 = 17, 24, 31, etc. So, if you go backwards, it is also congruent to 3-7 = -4 (the remainder), 3-7-7 = -11, -18, -25, etc. Some people think you can use the remainder operation to get a positive modulus (3), but that’s not how it works for negative numbers

http://math.stackexchange.com/questions/801962/difference-between-modulus-and-remainder
http://stackoverflow.com/questions/13683563/whats-the-difference-between-mod-and-remainder

As we know from grade school, when we divide one integer by another (nonzero) integer we get an integer quotient (the "answer") plus a remainder (generally a rational number). For instance,
13/5 = 2 ("the quotient") + 3/5 ("the remainder").
We can rephrase this division, totally in terms of integers, without reference to the division operation:
13 = 2(5) + 3.
Note that this expression is obtained from the one above it by multiplying through by the divisor 5.
We refer to this way of writing a division of integers as the Division Algorithm for Integers. More formally stated:

If a and b are positive integers, there exist integers unique non-negative integers q and r so that
a = qb + r , where 0[less than or equal]r < b.
q is called the quotient and r the remainder.

% GCD
The greatest common divisor of integers a and b, denoted by gcd(a,b), is the largest integer that divides (without remainder) both a and b. So, for example:

gcd(15, 5) = 5,	gcd(7, 9) = 1,	gcd(12, 9) = 3,	gcd(81, 57) = 3.

It is well known that if the gcd(a, b) = r then there exist integers p and s so that:

p(a) + s(b) = r.
By reversing the steps in the Euclidean Algorithm, it is possible to find these integers p and s. We shall do this with the above example:
Starting with the next to last line, we have:

3 = 9 -1(6)
From the line before that, we see that 6 = 24 - 2(9), so:
3 = 9 - 1(24 - 2(9)) = 3(9) - 1(24).
From the line before that, we have 9 = 57 - 2(24), so:
3 = 3( 57 - 2(24)) - 1(24) = 3(57) - 7(24).
And, from the line before that 24 = 81 - 1(57), giving us:
3 = 3(57) - 7( 81 - 1(57)) = 10(57) -7(81).
So we have found p = -7 and s = 10.

The procedure we have followed above is a bit messy because of all the back substitutions we have to make. It is possible to reduce the amount of computation involved in finding p and s by doing some auxillary computations as we go forward in the Euclidean algorithm (and no back substitutions will be necessary). This is known as the extended Euclidean Algorithm.

%http://www.math.utah.edu/~fguevara/ACCESS2013/Euclid.pdf
%http://pages.pacificcoast.net/~cazelais/222/xeuclid.pdf

%https://www.khanacademy.org/computing/computer-science/cryptography/modarithmetic/a/the-euclidean-algorithm

http://www.cut-the-knot.org/recurrence/conversion.shtml

http://math.stackexchange.com/questions/48968/how-to-change-from-base-n-to-m

http://www.cut-the-knot.org/blue/frac_conv.shtml

http://cs.stackexchange.com/questions/10318/the-math-behind-converting-from-any-base-to-any-base-without-going-through-base
https://en.wikipedia.org/wiki/Positional_notation#Base_conversion
https://en.wikipedia.org/wiki/Binary_number#Conversion_to_and_from_other_numeral_systems

\begin{ExerciseList}

\Exercise Reduce the following fractions
\Question $88/72$
\Question $\frac{2}{\frac{4}{5}}$
\Question $\frac{\frac{-7}{2}}{\frac{4}{9}}$
\Answer Canceling out common terms we get
\begin{enumerate}
\item\myindent $88/72 = (11*8)/(9*8) = 11/9$.
\item\myindent $\frac{2}{\frac{4}{5}} = 2 \cdot \frac{5}{4} = \frac{10}{4} = \frac{5}{2}$
\item\myindent $\frac{\frac{-7}{2}}{\frac{4}{9}} = \frac{-7}{-2} \cdot \frac{9}{4} = \frac{63}{8}$
\end{enumerate}

\Exercise If it takes 36 minutes for $5$ people to paint $9$ walls. How many minutes does it take $10$ people to paint $7$ walls?
\Answer $36/9$ is the time it takes for $5$ people to paint one wall, therefor $5 * 36/9$ is the time it takes for one person to paint one wall. Thus it will take $7 * ((5 * 36)/9) = 140$ for one person to paint $7$ walls and therefor $140/10 = 14$ minutes for $10$ people

\Exercise Philip is going on a $4000$-kilometer trip with his three friends. The car consumes $6$ liters per $100$ kilometers, and gas costs $1.50$ per liter. how much should each pay if they want to split the cost of gas evenly?
\Answer The car will spend $4000/100 * 6 = 240$ liters of gas at a price of $240 * 1.5 = 360$, as there are four people in the car the cost per person becomes $360/4 = 90$.

% TODO add to Solvr arithmetic tests
\Exercise What is the value of
\Question $7 + 7/7 + 7 \cdot 7 - 7$
\Question $6 -1 \cdot 0 + 2/2$
\Question $-2 + (-3) + 4 - (-3) - 5$
\Question $3 + (-4) - 8 - (-1) - 1$
\Answer Using the PEMDAS meme (parentheses, exponents, multiplication, division, addition, subtraction) listed in \ref{arit:order} we get:
\begin{enumerate}
 \item\myindent $7 + 7/7 + 7 cdot 7 - 7 = 7 + (7/7) + (7 \cdot 7) - 7 = 50$.
 \item\myindent $6 - 1 \cdot 0 + 2/2 = 6 + -1 \cdot 0 + 2/2 = 6 + 0 + 2/2 = 7$
 \item\myindent $-2 + (-3) + 4 - (-3) - 5 = -2 - 3 + 4 + 3 - 5 = -5 + 7 - 5 = -3$
 \item\myindent $3 + (-4) - 8 - (-1) - 1 = 3 - 4 - 8 + 1 - 1 = -1 - 8 $
\end{enumerate}

\Exercise Calculate the roots 
\Question $x^2 = 144$
\Question $x^3 = 216$
\Answer Using prime factorisation and the long hand square root algorithm we get
\begin{enumerate}
\item\myindent $\sqrt[2]{x^2} = \sqrt[2]{144}$ factoring gives $144 = 16 \times 9 = 3^2 \times 2^4 = 12 \times 12$ so $x = 12$.
\item\myindent $\sqrt[3]{x^3} = \sqrt[3]{216}$ factoring gives $216 = 8 \times 27 = 2^3 \times 3^3 = 6 \times 6 \times 6$ so $x = 6$.
\end{enumerate}

\Exercise Round $838.274$ to the
\Question Nearest hundred
\Question Nearest ten 
\Question Nearest one
\Question Nearest tenth
\Question Nearest hundredth
\Answer Using the rounding algorithm from \ref{arit:rounding} we get
\begin{enumerate}
\item\myindent $800$ as $3 < 5$
\item\myindent $840$ as $8 > 5$
\item\myindent $838$ as $2 < 5$
\item\myindent $838.3$ as $7 > 5$
\item\myindent is $838.27$ as $4 < 5$
\end{enumerate}

\Exercise Reduce the following
\Question Rewrite $\frac{4^6}{4^{-4}}$ in the form $4^n$
\Question Rewrite $(5^{-12})^{-9}$ in the form $5^n$
\Question Rewrite $((5^{-10})(9^9))^7$ in the form $5^n \cdot 9^m$
\Question Rewrite $\sqrt{11} + \sqrt{44} + \sqrt{99}$ in the form $n \cdot \sqrt{m}$
\Answer Using the exponent rules from \ref{arit:exp} we get
\begin{enumerate}
\item\myindent $\frac{4^6}{4^{-4}} = 4^{6 - -4} = 4^{10}$
\item\myindent $(5^{-12})^{-9} = 5^{108}$
\item\myindent $((5^{-10})(9^9))^7 = 5^{-70} \cdot 9^{63}$
\item\myindent $\sqrt{11} + \sqrt{44} + \sqrt{99} =
                \sqrt{11} + \sqrt{4 \cdot 11} + \sqrt{9 \cdot 11} = 6\sqrt{11}$
\end{enumerate}
\end{ExerciseList}
