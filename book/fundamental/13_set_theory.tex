\chapter{Set theory}
% TODO Sets missing concepts
% - union, complement and intersection of sets
% - filtered set, unordered sets, ordered set
%\section{Naive set theory}
%\section{Counting the Infinite}
%\subsection{Problems with naive set theory}
%\section{Axiomatic set theory}
%\section{Cantor and the Transfinite Realm}
% Morphism In many fields of mathematics, morphism refers to a structure-preserving mapping[disambiguation needed] from one mathematical structure to another.
% Set builder notation
% Cartesian product (including algorithm)
% Power set

Loosely speaking a set is a collection of well defined objects. 

- Properties
    - Cardinality - number of elements in the set
    - X in y
    - X !in y
    - Sets can contain other sets
    - Two sets are equal if they contain the same elements {1,2} = {2,1,2} (order and repetitions are not important)
    - x subset y - every element of x is in y
    - x proper-subset y - every element of x is in y and x is not equal to y
    - x union y = x in X or x in Y (or x in both)
    - x intersect y = x in X and x in Y
    - x \ y = x in X and x !in y (if y subset x then x\y is complement of y in x)
    - (a,b) - interval greater than a and less than b
    - [a,b] - interval from a to b (both included)
    - X x Y - set product is all pairs (x,y) where x in X and y in Y

Some common operations on sets
\begin{description}
\item[Empty set] $\varnothing = \{\}$.
\item[Set intersection] $S \cap T$ is the set containing all the elements that are in both $S$ and $T$:
\begin{equation}
S \cap T := \left\{{x: x \in S \land x \in T}\right\}
\end{equation}
\item[Set union] $S \cup T$ is the set containing all the elements that are in either or both of the sets $S$ and $T$:
\begin{equation}
S \cup T := \left\{{x: x \in S \lor x \in T}\right\}
\end{equation}
\end{description}



A indexed set is a collection of values associated with indices. For example
\begin{itemize}
\item An ordered pair is a family indexed by the two element set $2 = \{1, 2\}$.
\item An n-tuple is a family indexed by $n$.
\end{itemize}
the set that whose members label (or index) members of a family is called an index set. 

- 


- Operations
    - Equivalence
    - Converse
    - Implication
- Functions
    - Map
    - Injective
    - Bijective
    - Surjective
    - Inverse
- Proof
    - Definition
    - Statement
    - Theorem


standard refresher on how to write a recursive method:

- Are we in a trivial case? Then solve the problem and return the solution.
- Otherwise, split the problem into one or more strictly smaller problems.
- Solve the subproblems recursively.
- Combine their solutions to solve the larger problem.

\section{Sets}


\section{Exercises}
\begin{ExerciseList}
\end{ExerciseList}
