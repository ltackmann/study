\chapter{Set theory}
% TODO Sets missing concepts
% - union, complement and intersection of sets
% - filtered set, unordered sets, ordered set
%\section{Naive set theory}
%\section{Counting the Infinite}
%\subsection{Problems with naive set theory}
%\section{Axiomatic set theory}
%\section{Cantor and the Transfinite Realm}
% Morphism In many fields of mathematics, morphism refers to a structure-preserving mapping[disambiguation needed] from one mathematical structure to another.
% Set builder notation
% Power set
%
%- Operations
%    - Equivalence
%    - Converse
%    - Implication
%- Functions
%    - Map
%    - Injective
%    - Bijective
%    - Surjective
%    - Inverse

\section{Sets}
Loosely speaking a set is a collection of well defined objects. 

    - Cardinality - number of elements in the set
    - X in y
    - X !in y
    - Sets can contain other sets
    - Two sets are equal if they contain the same elements {1,2} = {2,1,2} (order and repetitions are not important)
    - x subset y - every element of x is in y
    - x proper-subset y - every element of x is in y and x is not equal to y
    - x union y = x in X or x in Y (or x in both)
    - x intersect y = x in X and x in Y
    - x \ y = x in X and x !in y (if y subset x then x\y is complement of y in x)
    - (a,b) - interval greater than a and less than b
    - [a,b] - interval from a to b (both included)
    - X x Y - set product is all pairs (x,y) where x in X and y in Y

Some common operations on sets
\begin{description}
\item[Empty set] $\varnothing = \{\}$.
\item[Set intersection] $S \cap T$ is the set containing all the elements that are in both $S$ and $T$:
\begin{equation}
S \cap T := \left\{{x: x \in S \land x \in T}\right\}
\end{equation}
\item[Set union] $S \cup T$ is the set containing all the elements that are in either or both of the sets $S$ and $T$:
\begin{equation}
S \cup T := \left\{{x: x \in S \lor x \in T}\right\}
\end{equation}
\item[Cardinality] For finite sets A and B, |A × B| = |A|·|B|. 
% TODO include algorithm
\end{description}

A indexed set is a collection of values associated with indices. For example
\begin{itemize}
\item An ordered pair is a family indexed by the two element set $2 = \{1, 2\}$.
\item An n-tuple is a family indexed by $n$.
\end{itemize}
the set that whose members label (or index) members of a family is called an index set. 

\section{Functions}
% TODO consider merging with input from "thinking like a mathematician"
Loosely speaking, a function is a rule that, for each element in some set D of possible inputs,  assigns a possible output. The output is said to be the image of the input under the function 

For a function named f, the image of q under f is denoted by f(q). If r = f(q), we say that  q maps to r under f. The notation for “q maps to r” is q )→ r. (This notation omits specifying  the function; it is useful when there is no ambiguity about which function is intended.)  It is convenient when specifying a function to specify a co-domain for the function. The  co-domain is a set from which the function’s output values are chosen. Note that one has some  leeway in choosing the co-domain since not all of its members need be outputs.  The notation  f : D −→ F  means that f is a function whose domain is the set D and whose co-domain (the set of possible  outputs) is the set F. (More briefly: “a function from D to F”, or “a function that maps D to F.”) 

Consider the function prod that takes as input a pair of integers greater than  1 and outputs their product. The domain (set of inputs) is the set of pairs of integers greater  than 1. We choose to define the co-domain to be the set of all integers greater than 1. The  image of the function, however, is the set of composite integers since no domain element maps  to a prime number. 

\subsection{Identity function} 
For any domain D, there is a function idD : D −→ D called the identity function for D, defined  by  idD(d) = d  for every d ∈ D. 

\subsection{Composition of functions}  
The operation functional composition combines two functions to get a new function. We will later  define matrix multiplication in terms of functional composition. Given two functions f : A −→ B  and g : B −→ C, the function g ◦f, called the composition of g and f, is a function whose domain  is A and its co-domain is C. It is defined by the rule  (g ◦ f)(x) = g(f(x))  for every x ∈ A.  If the image of f is not contained in the domain of g then g ◦ f is not a legal expression.  Example 0.3.10: Say the domain and co-domains of f and g are R, and f(x) = x + 1 and  g(y) = y2. Then g ◦ f(x)=(x + 1)2. 

show that composition of functions is associative:  

Proposition 0.3.12 (Associativity of composition): For functions f, g, h,  h ◦ (g ◦ f)=(h ◦ g) ◦ f 
% TODO proof
Let x be any member of the domain of f.  
(h ◦ (g ◦ f))(x) = h((g ◦ f)(x)) by definition of h ◦ (g ◦ f))  
                 = h(g(f(x)) by definition of g ◦ f  
                 = (h ◦ g)(f(x)) by definition of h ◦ g  
                 = ((h ◦ g) ◦ f)(x) by definition of (h ◦ g) ◦ f 
                 
functional inverse of h as well.  In general,  Definition 0.3.13: We say that functions f and g are functional inverses of each other if  • f ◦ g is defined and is the identity function on the domain of g, and  • g ◦ f is defined and is the identity function on the domain of f.  Not every function has an inverse. A function that has an inverse is said to be invertible.  Examples of noninvertible functions are shown in Figures 2 and 3  Definition 0.3.14: Consider a function f : D −→ F. We say that f is one-to-one if for every  x, y ∈ D, f(x) = f(y) implies x = y. We say that f is onto if, for every z ∈ F, there exists  x ∈ D such that f(x) = z. 

\section{Exercises}
\begin{ExerciseList}
% TODO rewrite sample
%\Exercise What is the cardinality of {1, 2, 3,..., 10, J, Q, K} × {♥, ♠, ♣, ♦}? 
%\Answer The cardinality of the first set is 13, and the cardinality of the  second set is 4, so the cardinality of the Cartesian product is 13 · 4, which is 52. 
\end{ExerciseList}
