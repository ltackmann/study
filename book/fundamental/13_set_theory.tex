\chapter{Sets and functions}
% - (a,b) - interval greater than a and less than b
% - [a,b] - interval from a to b (both included)
% - X x Y - set product is all pairs (x,y) where x in X and y in Y

Sets and functions are the basic building blocks of most mathematics and while they might appear so trivial to not warrant any discussion, getting a good grip of their behaviour is essential to becoming skilled at mathematics (and indeed for reading and understanding this book). Loosely speaking a set is a collection of objects, and a function is an association of members of one set to members of another. 

\section{Sets}
A set is a collection of well defined objects. We use the notation $\{1,2,3\}$ to denote the set containing the elements $one$, $two$ and $three$. For sets we don't care about their order nor about duplicates so $\{1,2,3\}$ is the same set as $\{3,2,1\}$ which is the same as $\{3,2,1,2,3\}$. If element $x$ is a member of the set $X$, we write $x \in X$, for example $3 \in \{1,2,3\}$. Similarly if $x$ is not a member, then we write $x \notin X$, for example $5 \notin \{1,2,3\}$. The reason why we need this rigor of defining membership of sets using $\in$ and $\notin$ is because for many mathematical results we need to be clear about what objects/numbers a particular theorem applies to in order to prove it.
 
\bigskip
Sets are not only restricted to numbers, their elements can be any well defined object. So we can have a set such as $\{\text{car}, \text{bike}\}$. Further sets can also contain other sets, such as the set $Y = \{1, 2, \{3\}\}$. Where $Y$ is the set containing the numbers $one$ and $two$ and the set $\{3\}$. Note the importance of the difference between being a set and being an element of a set. In the previous example $1 \in Y$ but $3 \notin Y$. The later is because its the set $\{3\}$ that is the member of $Y$ and not the number $three$ that is a member. That is $\{3\} \in Y$ as we must distinguish between $3$ and $\{3\}$.

\subsection{Some common sets and operations}
In the following we will introduce some common classes (sets) of numbers. We will use these to demonstrate how we can work with sets, some common operations on sets and why clear definitions of set membership is important. The first set of numbers we will introduce is the most basic, counting (or natural numbers):
\begin{definition}
The natural numbers is the whole, non-negative numbers $\{1,2,3, \cdots\}$ and is denoted by $\mathbb{N}$. These are named "natural" as their members represent natural occurring quantities that can be counted such as "two birds" or "one apple", for this reason the natural numbers are often referred to as the counting numbers.  
\end{definition}
Note the number $0$ is usually not included in the set of natural numbers such that $0 \notin  \mathbb{N}$ however some mathematicians does indeed include zero even though its not a natural occurring quantity as it can be a helpful in certian mathematical proofs to have it included. Once we have the counting numbers in place it becomes natural to extend them with negative numbers, to be able to denote concepts such as a debt owed or the subtraction of a quantity from another. We therefor define the integers to be the set of numbers containing both the natural numbers as well as the negative whole numbers (and zero). 
\begin{definition}
The set of integers is $\{\cdots −3, −2, 0, 1, 2, 3\cdots\}$ and is denoted by the symbol $\mathbb{Z}$. 
\end{definition}
As we can see the natural numbers is contained within the set of integers. This property that every element in one set is contained in another set is known as being a subset and written as $\mathbb{N} \subseteq \mathbb{Z}$. More formally we can define a subset as
\begin{definition}
The set $Y$ is a subset of the set $X$ if every element of $Y$ is also an element of $X$. That is we write $Y \subseteq X$ if for all $x \in Y$, then $x \in X$. 
\end{definition}
Note that the reverse is not true, that is the the integers is not a subset of the natural numbers so that $\mathbb{Z} \nsubseteq \mathbb{N}$. 

\bigskip
Examples
(i) The set Y = {1, {3, 4}, mouse} is a subset of X = {1, 2, dog, {3, 4}, mouse}. 
(ii) The set of even numbers is a subset of N.
(iii) The set {1, 2, 3} is not a subset of {2, 3, 4} or {2, 3}. 
(iv) For any set X, we have X ⊆ X. 
(v) For any set X, we have ∅ ⊆ X. 
Remark 1.13 It is vitally important that you distinguish between being an element of a set and being a subset of a set. These are often confused by students. If x ∈ X, then {x} ⊆ X. Note the brackets. Usually, and I stress usually, if x ∈ X, then {x} ∉ X, but sometimes {x} ∈ X, as the following special example shows. 
% TODO create into an excercise
Example 1.14 Consider the set X = {x, {x}}. Then x ∈ X and {x} ⊆ X (the latter since x ∈ X) but we also have {x} ∈ X. Therefore we cannot state any simple rule such as ‘if a ∈ A, then it would be wrong to write a ⊆ A’, and vice versa.

\bigskip

Propper subset
A subset Y of X is called a proper subset of X if Y is not equal to X. We denote this by Y ⊂ X. Some people use Y  X for this. 
Examples 1.16 
(i) {1, 2, 5} is a proper subset of {−6, 0, 1, 2, 3, 5}. 
(ii) For any set X, the subset X is not a proper subset of X. 
(iii) For any set X ≠ ∅, the empty set ∅ is a proper subset of X. Note that, if X = ∅, then the empty set ∅ is not a proper subset of X.
(iv) For numbers, we have N ⊂  ⊂ Q ⊂  ⊂ C.
Now let’s consider where the notation came from. It is is obvious that for a finite set the two statements If X ⊆ Y, then |X| ≤ |Y |, and If X ⊂ Y, then |X| < |Y| are true. So ⊆ is similar to ≤ and ⊂ is similar to < as concepts and not just as symbols.

From this set it is easy to define the non-negative integers, {0, 1, 2, 3, 4, . . . }, often denoted Z+. Note that all natural numbers are integers.
\bigskip

Rational numbers
The set of rational numbers is denoted by Q and consists of all fractional numbers, i.e. x ∈ Q if x can be written in the form p/q where p and q are integers with q ≠ 0. For example, 1/2, 6/1 and 80/5. Note that the representation is not unique since, for example, 80/4 = 40/2 = 20/1 = 20 Note also that all integers are rational numbers since we can write x ∈ Z as x/1.

Real numbers
The real numbers, denoted R, are hard to define rigorously. For the moment let us take them to be any number that can be given a decimal representation (including infinitely long representations) or as being represented as a point on an infinitely long number line. The real numbers include all rational numbers (hence integers, hence natural numbers). Also real are π and e, neither of which is a rational number. 4 The number √2 is not rational as we shall see in Chapter 23. The set of real numbers that are not rational are called irrational numbers.

complex numbers
denoted , by pretending that the square root of −1 exists. This is one of the most powerful additions to the mathematician’s toolbox as complex numbers can be used in pure and applied mathematics.

Transendential numbers
% TDOO you have written about this elsewhere

\subsection{Properties of sets}
% Go through numbers and use them to define various properties such as 
% Set builder notation
% Power set
% - Cardinality - number of elements in the set
% - X in y
% - X !in y
% - Sets can contain other sets
% - x subset y - every element of x is in y
% - x proper-subset y - every element of x is in y and x is not equal to y
% x union y = x in X or x in Y (or x in both)
% - x intersect y = x in X and x in Y
% - x \ y = x in X and x !in y (if y subset x then x\y is complement of y in x)


Empty set
The set with no elements is called the empty set and is denoted ∅. It may appear to be a strange object to define. The set has no elements so what use can it be? Rather surprisingly this set allows us to build up ideas about counting.
$\varnothing = \{\}$
The set {∅} is the set that contains the empty set. This set has one element. Note that we can then write∅ ∈ {∅}, but we cannot write ∅ ∈ ∅ as the empty set has, by definition, no elements.

Equality
Two sets are equal if they have the same elements. If set X equals set Y then we write X = Y . If not we write X ≠ Y.
% TODO as examples and excercises
(i) The sets {5, 7, 15} and {7, 15, 5} are equal, i.e. {5, 7, 15} = {7, 15, 5}. 
(ii) The sets {1, 2, 3} and {2, 3} are not equal, i.e. {1, 2, 3} ≠ {2, 3}. 
(iii) The sets {2, 3} and {{2}, 3} are not equal. 
(iv) The sets R and N are not equal.

Cardinality
If the set X has a finite number of elements, then we say that X is a finite set. If X is finite, then the number of elements is called the cardinality of X and is denoted |X|. For finite sets A and B, |A × B| = |A|·|B|.
If X has an infinite number of elements, then it becomes difficult to define the cardinality of X. We shall see why in Chapter 30. Essentially it is because there are different sizes of infinity! 
% TODO refer to section on cantor
% TODO include algorithm
Examples 
(i) The set {∅, 3, 4, cat} has cardinality 4. 
(ii) The set {∅, 3, {4, cat} } has cardinality 3.
Exercises 
What is the cardinality of the following sets? 
(i) {1, 2, 5, 4, 6} 
(ii) {π, 6, {π, 5, 8, 10}} 
(iii) {π, 6, {π, 5, 8, 10}, {dog, cat, {5}}}
(iv) ∅ 
(v) N 
(vi) {dog, ∅} 
(vii) {∅, {∅, {∅}}} 
(viii) {∅, {20, π, {∅}}, 14}





% TODO Sets missing concepts
% - union, complement and intersection of sets
% - filtered set, unordered sets, ordered set

\begin{description}
\item[Set intersection] $S \cap T$ is the set containing all the elements that are in both $S$ and $T$:
\begin{equation}
S \cap T := \left\{{x: x \in S \land x \in T}\right\}
\end{equation}
\item[Set union] $S \cup T$ is the set containing all the elements that are in either or both of the sets $S$ and $T$:
\begin{equation}
S \cup T := \left\{{x: x \in S \lor x \in T}\right\}
\end{equation} 

\end{description}

A indexed set is a collection of values associated with indices. For example
\begin{itemize}
\item An ordered pair is a family indexed by the two element set $2 = \{1, 2\}$.
\item An n-tuple is a family indexed by $n$.
\end{itemize}
the set that whose members label (or index) members of a family is called an index set. 

\section{Functions}
%- Functions
%    - Map
%    - Injective
%    - Bijective
%    - Surjective
%    - Inverse

% TODO consider merging with input from "thinking like a mathematician"
Loosely speaking, a function is a rule that, for each element in some set D of possible inputs,  assigns a possible output. The output is said to be the image of the input under the function 

For a function named f, the image of q under f is denoted by f(q). If r = f(q), we say that  q maps to r under f. The notation for “q maps to r” is q )→ r. (This notation omits specifying  the function; it is useful when there is no ambiguity about which function is intended.)  It is convenient when specifying a function to specify a co-domain for the function. The  co-domain is a set from which the function’s output values are chosen. Note that one has some  leeway in choosing the co-domain since not all of its members need be outputs.  The notation  f : D −→ F  means that f is a function whose domain is the set D and whose co-domain (the set of possible  outputs) is the set F. (More briefly: “a function from D to F”, or “a function that maps D to F.”) 

Consider the function prod that takes as input a pair of integers greater than  1 and outputs their product. The domain (set of inputs) is the set of pairs of integers greater  than 1. We choose to define the co-domain to be the set of all integers greater than 1. The  image of the function, however, is the set of composite integers since no domain element maps  to a prime number. 

\subsection{Identity function} 
For any domain D, there is a function idD : D −→ D called the identity function for D, defined  by  idD(d) = d  for every d ∈ D. 

\subsection{Composition of functions}  
The operation functional composition combines two functions to get a new function. We will later  define matrix multiplication in terms of functional composition. Given two functions f : A −→ B  and g : B −→ C, the function g ◦f, called the composition of g and f, is a function whose domain  is A and its co-domain is C. It is defined by the rule  (g ◦ f)(x) = g(f(x))  for every x ∈ A.  If the image of f is not contained in the domain of g then g ◦ f is not a legal expression.  Example 0.3.10: Say the domain and co-domains of f and g are R, and f(x) = x + 1 and  g(y) = y2. Then g ◦ f(x)=(x + 1)2. 

show that composition of functions is associative:  

Proposition 0.3.12 (Associativity of composition): For functions f, g, h,  h ◦ (g ◦ f)=(h ◦ g) ◦ f 
% TODO proof
Let x be any member of the domain of f.  
(h ◦ (g ◦ f))(x) = h((g ◦ f)(x)) by definition of h ◦ (g ◦ f))  
                 = h(g(f(x)) by definition of g ◦ f  
                 = (h ◦ g)(f(x)) by definition of h ◦ g  
                 = ((h ◦ g) ◦ f)(x) by definition of (h ◦ g) ◦ f 
                 
functional inverse of h as well.  In general,  Definition 0.3.13: We say that functions f and g are functional inverses of each other if  • f ◦ g is defined and is the identity function on the domain of g, and  • g ◦ f is defined and is the identity function on the domain of f.  Not every function has an inverse. A function that has an inverse is said to be invertible.  Examples of noninvertible functions are shown in Figures 2 and 3  Definition 0.3.14: Consider a function f : D −→ F. We say that f is one-to-one if for every  x, y ∈ D, f(x) = f(y) implies x = y. We say that f is onto if, for every z ∈ F, there exists  x ∈ D such that f(x) = z. 

\section{Exercises}
\begin{ExerciseList}
% TODO rewrite sample
%\Exercise What is the cardinality of {1, 2, 3,..., 10, J, Q, K} × {♥, ♠, ♣, ♦}? 
%\Answer The cardinality of the first set is 13, and the cardinality of the  second set is 4, so the cardinality of the Cartesian product is 13 · 4, which is 52. 
\end{ExerciseList}
