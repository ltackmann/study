\chapter{Analytical geometry}
% TODO missing analythical geometry concepts
% - describe structure of coordinate system
% - describe connections between functions and ordered pair (x, f(x))
% - recreate plane and solids

Analytic geometry, also known as coordinate geometry, or Cartesian geometry, is the study of geometry using a coordinate system.

\section{Coordinate systems}
Coordinate systems are used to uniquely define the position of a point or other geometric or physical object. An everyday use of a coordinate system is the grid reference used to locate places on a map.

% TODO draw picture of map

\subsection{Cartesian coordinate system}
A plane is a flat two-dimensional surface, so Cartesian coordinates in two dimensions are often known as the Cartesian plane.

We define the position of a point, P for example, in terms of its x, y and z coordinates. So if P's position is at x = 2, y = 3, z =4 we say (2,3,4) are the coordinates of P. The coordinates of various points on the Cartesian plane are shown in Figure

% TODO draw two dimensional cartesian coordinate system

The x, y and x, y, z coordinates we've used so far, where the axes (singular - axis, as in ‘the x axis’) intersect at right angles, are known as the Cartesian or rectangular coordinate system, with the point O where the axes intersect called the origin.

% TODO draw tree dimensional cartesian coordinate system

\subsection{Polar coordinate system}
Although Cartesian coordinates are a nice, simple coordinate system, they aren't so suitable when circular, cylindrical, or spherical symmetry is present. In those circumstances, in two dimensions, plane polar coordinates are a better choice, where the position of a point on the plane is given in terms of the distance r from a fixed point and an angle θ from a fixed direction (see Figure 1.10). So, if point P was a distance r = 6 from the origin, and the angle θ = 120°, the coordinates of P would be (6, 120°).

% TODO draw example and show how to convert between coordinate systems

\subsection{Spherical coordinate system}
In three dimensions, the position of a point P in space can be defined using spherical coordinates (r, θ, ϕ) as shown in Figure 1.12. To go from spherical coordinates to Cartesian coordinates we see that

% TODO draw example and show how to convert between coordinate systems

% EXCERCISES
% - Graph the circle (x+3)^2+(y+2)^2=9.

