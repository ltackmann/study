\chapter{Numerical analysis}
% TODO missing numerical concepts
% - Compare with Archimedes calculations of PI
% - MCMC (Markov chain Monte Carlo) https://en.wikipedia.org/wiki/Metropolis–Hastings_algorithm
% - Babylonian method http://en.wikipedia.org/wiki/Methods_of_computing_square_roots#Babylonian_method
% - Runge–Kutta for ordinary differential eqautions http://en.wikipedia.org/wiki/Runge–Kutta_methods
% - Newton rapson method for square roots http://www.sosmath.com/calculus/diff/der07/der07.html
% - Bisection method https://en.wikipedia.org/wiki/Bisection_method
% - Fast inverse square root https://en.wikipedia.org/wiki/Fast_inverse_square_root
% - Finite difference methods (check "Finite difference methods in Finance" book on OneFrive)
% - Trapez integration
Already the Babylonians knew how to approximate square roots, their method was based on simple process of continuously improving guesses. For example to find $\sqrt{10}$ we may guess $3$ then square it to get $9$, as this is less than 10 we know $3 < \sqrt{10}$ next we may guess $4$ which squares $16$ as this is above $10$ we know now $3 < \sqrt{10} < 4$. For our next guess we may take the mean of the boundaries $3$ and $4$ resulting in $(1/2 \times (3+4))^2 = (7/2)^2 = 49/4$ as $49/4 > 10$ we now know that $3 < \sqrt{10} 7/2$. Applying the last step again we find $(1/2 \times (3 + 7/2))^2 = (13/4)^2 = 169/16$ which again is above $10$ so now we know $3 < \sqrt{10} 13/4$ and so on.

However this method seam crude and tedious to carry out by hand, to improve it we need to get better at guessing which numbers are good candidates for approximations of the square root of a number $n$. To do this lets examine the nature of $\sqrt{n}$. Informally if we assume $(n/2)^2 < n$ for $n>0$ then
\[
(n/2)^2 < n \rightarrow n^2/4 < n \rightarrow n < 4
\]
so for positive numbers less than $4$ our initial guess has to be above half $n$ (the approximations lower bound is $n/2$). Conversely if we assume $(n/2)^2 > n$ for $n>0$ then
\[
(n/2)^2 > n \rightarrow n^2/4 > n \rightarrow n > 4
\]
S
That is for numbers above $4$ our initial guess has to be below half $n$ (the approximations upper bound is $n/2$).

\section{The Newton-Raphson Method}
The Newton-Raphson Method

\subsection{Roots}

\section{Approximating differential equations}
\subsection{Finite-difference methods}

\section{Numerical integration}

\section{Exercises}
\begin{ExerciseList}
\end{ExerciseList}
