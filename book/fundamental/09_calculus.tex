\chapter{Calculus}
% TODO missing calculus concepts
% http://www.science4all.org/le-nguyen-hoang/infinite-series/
% - Area under curve
% - Fundamental theorem of calculus
% - Proof and explain taylor series
% - Proof Finite geometric series https://www.khanacademy.org/math/precalculus/seq_induction/geometric-sequence-series/v/geometric-series-introduction
% - Numerical integration http://rosettacode.org/wiki/Numerical_integration#ActionScript
% - LaplaceTransform http://reference.wolfram.com/language/ref/LaplaceTransform.html
% - story of e (proof the limit of the series)
% - relationship between pi, e and i
% - include info on e from "the story of e"
% - include info on i from  "the story of i"
% Gamma function and Stirling's approximation
% TODO eulers identity
% general binomial theorem (the binomial theorem for fractional and negative powers).
% https://medium.com/fintechexplained/calculus-2-a-must-know-concept-for-every-professional-a86599caa2a5

% TODO rewrite
Differentiation: Helps us understand how an output of a function changes when its inputs change. This article will concentrate on the concept of differentiation.
Integration: Helps us calculate the total impact of the change over time and helps us compute a function from its derivative. The next article will outline the concept of integration.
Rate of change is a measure of how much a quantity changes (increases or decreases) in relation to the change in another quantity.
Derivative of a function measures how sensitive an output of a function is to the input of the function. Differentiation is the name of the process in which a set of steps are performed to find the derivative of a function.

Example
Let’s consider that a variable y depends on a variable x. For instance, y could be the speed of a football player kicking a ball and x could be the number of hours the player spent in the training camp.
Let’s now start recording the ball speed and the number of hours the players spent in the training camp. Let’s also record whether they are left footed or right footed by the sign of x: negative sign implies that the player is left footed and position sign means that the player is right footed:
When x=-4, y =16.
When x =-3, y =9
When x=0, y =0.
When x =1, y =1.
When x =2, y = 4.
When x =3,
When x= 10, y = 100
When x = -11, y = 121
If we plot the chart where y points are plotted on the y axis and x points are plotted on the x-axis then we’ll see a y = x² chart:

We could pick a point on the chart (x,y), draw a line tangent to the chart and take our rate of change as change in y over change in x. We would then repeat this exercise for all points on the chart and calculate the average of the gradient.
Tangent is a strange line that aims to have the same gradient as the curve.

Rate of change (or gradient) = difference in y / difference in x = dy/dx

Slope is also known as the gradient of a chart

The gradient at point (x1,y1) and (x2,y2) will be completely different from the gradient at point (y4,x3) and (x2, y2).
We could instead use the techniques of calculus and utilise differentiation to find the instantaneous rate of change. Differentiation will enable us to find the rate of change of value of y when the value of x changes by a very small unit. Derivative is essentially finding the rate of change in the function when we change the value of x.

We understood that derivative is all about finding how fast/slow the output of a function changes when we change the input of the function. Therefore, we can start changing the value of x by 1 unit and start seeing how the value of y changes. Or, we could plot the chart and pick any of the two points on x axis, then find the value of y and then calculate the slope.
Let’s refer to the small increment to the input in x as Z.
Therefore, the function f(x) becomes: f(x + Z) implying that we are adding Z to the value of x.

rate of change (or gradient) = (f(x + Z) - f(x))/(x+Z-x)

This is known as Newton’s difference quotient.

the power rule.
d/dx x^a = a * x^(a-1)
Therefore: If f(x) = x² Then its derivative is 2x

Second Order Differentiation:
It merely means repeating the differentiation twice. Differentiating a function returns a new function. Second order differentiation is about calculating the derivative of the derivative:

% Excercises
% - derivatie of x^3 = 3x^2
% - derivatie of 3x^2 = 6x

\section{Logarithms}
If I'm driving at a constant velocity, say 60 km per hour, the distance I travel is changing in a constant manner, ie every hour I cover 60 km. We can say the rate of change of the distance I travel with respect to time is constant. However, if my car is moving with constant acceleration, its velocity is changing in a constant manner, but the distance I travel is not changing constantly - because I'm accelerating. We can say the rate of change of the distance I travel, with respect to time, is not constant.

\section{Differential calculus}
Differential calculus allows us to find the exact slope of a curve, in other words a function's rate of change, at any particular point. We are assuming that the function is differentiable in the first place and that the gradient doesn't go shooting off to infinity.
\[
\frac{f(x + \Delta x) - f(x)}{\Delta x}
\]
is known as the difference quotient, or the Newton quotient, and its limit defines the derivative of a function.


\subsection{The derivative}
This limit is known as the derivative and is the heart of differential calculus. In this example, where $y = x^2$ the derivative could be denoted by
\[
\frac{dy}{dx}
\]

\[
f(x) = x^k 
f'(x) = kx^{k-1}
\]

\subsubsection{the product rule}
We use this rule when differentiating a product of two or more functions.

\subsubsection{the chain rule}
The chain rule This rule is used if a function is a function of another function. So, if y is a function of u and u is a function of x, then the derivative of y with respect to x is equal to the derivative of y with respect to u multiplied by the derivative of u with respect to x,

\section{Integral calculus}
For example, say we have a function that tells us the acceleration of an object after a certain time. If we can integrate that function we obtain a new function that tells us the velocity of the object after a certain time. If we integrated the second function, we'd get a third function that tells us the distance covered after a certain time. But we can also do that sequence of calculations in reverse, by starting with the function that tells us distance covered after a certain time. We can differentiate that function to find the object's velocity and differentiate again to find its acceleration.

% TODO explain why distance, velocity and accelaration is so inter linked (explain physical formulas)
% Hermite’s method
\subsection{Laplace Transform}


\section{Sequences}
A \index{Sequences}{sequence} is a ordered list of things (usually numbers) for example $\left\{1, 3, 5, 7\right\}$ is the finite sequence of the first $4$ odd numbers. and $\left\{1, 2, 4, 8, 16, 32, ...\right\}$ is an infinite sequence where every term doubles. When the sequence goes on forever it is called an infinite sequence, otherwise it is a finite sequence. Sequence is like a \index{Set}{set} except the terms are in order (with sets the order does not matter) and the same value can appear many times (only once in Sets). For example $\left\{0, 1, 0, 1, 0, 1, ...\right\}$ is the sequence of alternating $0$s and $1$s, the set is just $\left\{0,1\right\}$.

To make it easier to use rules, we often use this special style:

\section{Series}
A \index{Series}{series} is, informally speaking, the sum of the terms of a sequence.

\subsection{Taylor series}
\begin{equation}
cos(\theta) = \sum_{n=0}^{\infty} \frac{(-1)^n \cdot \theta^{2n}}{2n!}
\end{equation}

\subsection{Geometric Series}
\begin{equation}
a + ar + ar^2 + ar^3 + \cdots + ar^{n-1} = \sum_{k=0}^{n-1}ar^k = a \frac{1-r^n}{1-r}
\end{equation}

% artimetic series http://www.mathsisfun.com/algebra/sequences-sums-arithmetic.html


\section{Special functions}
\subsection{Logarithm}
A logarithm (log or logs for short) goes in the opposite direction to a power by asking the question: what power produced this number? So, if 2x = 32, we are asking, ‘what is the logarithm of 32 to base 2?’ We know the answer: it's 5, because 25 = 32, so we say the logarithm of 32 to base 2 is 5. In general terms, if , then we say the logarithm of a to base x equals p, or For example, 104 = 10,000, so we say the logarithm of 10,000 to base 10 equals 4, or We can take logarithms of any positive number, not just whole ones. So, as 103.4321 = 2704.581 we say the logarithm of 2704.581 to base 10 equals 3.4321, or Logarithms to base 10 are called common logarithms. Older readers may remember, many years ago - after the dinosaurs, but before calculators and computers were widely available - doing numerical calculations laboriously by hand using tables of common logarithms and anti-logarithms. The properties of logarithms are based on the aforementioned rules for working with powers. Assuming that a > 0 and b > 0 we can say: logx(ab) = logxa + logxb, eg log10(1000 × 100) = log10(100,000) = 5 = log10(1000) + log10(100) = 3 + 2. logx(1/a) = -logxa, eg log3(1/27) = -log327 = -3. logx(a/b) = logxa - logxb, eg log2(128/8) = log2(16) = 4 = log2(128) - log2(8) = 7 - 3. logx(ay) = ylogxa, eg log5(253) = log515,625 = 6 = 3 × log525 = 3 × 2. 1.8.3

% TODO algorithms for calculating logarithms (Taylor series) http://en.wikipedia.org/wiki/Logarithm#Power_series

\subsection{Natural logarithm}
% TODO plot y = ln{x} and add as compound ingerest in index (use correct symbols as used in compound intersts)
% the area under the hyperbola y = 1/x —led independently to the same number, leaving the exact origin of e shrouded in mystery.
Say we invest 1 in a bank that pays 100\% interest per year. If the bank calculated and credited the interest at the end of one year, our investment would then be worth $1 + 1 = 2$. But if the bank credits the interest more frequently than once a year say every six months, at the end of that period the balance would equal $(1 + 1/2)$ since we would get half of the yearly interest payed each six months. At the end of one year the total amount would be $(1 + 1/2)(1 + 1/2) = (1 + 1/2)^2$ as the interest (compund or basic). Calculated three times a year after four moths we would have $(1 + 1/3)$ after eight $(1 + 1/3)^2$ and the final balance would be $(1 + 1/3)^3$. In general, if interest is calculated n times a year, the balance after one year is
\[
f(k) = \left(1 + \frac{1}{k}\right)^k
\]
if we plug in some values we observe
\[
\begin{align*}
  f(2)  =& \left(1 + \frac{1}{2}\right)^2 = 2.25 \\
  f(5)  =& \left(1 + \frac{1}{5}\right)^5 = 2.248832 \\
  f(10) =& \left(1 + \frac{1}{10}\right)^{10} = 2.59374...
\end{align*}
\]

n 
1 2 
2 2.25 
3 2.37037 
4 2.44141 
5 2.48832 
10 2.59374 
100 2.70481 
1000 2.71692 
100,000 2.71827 
1,000,000 2.71828 
10,000,000 2.71828

We can see that as n increases, the value of the function appears to settle down to a number approximately equal to 2.71828. It can be shown that as n becomes infinitely large, it does indeed equal the constant e. The mathematically succinct way of saying this introduces the important idea of a limit and we say where means the limit of what follows (ie ) as n approaches infinity (symbol ). In other words, e approaches the value of as n approaches infinity.

\begin{equation}
e = \lim_{k\to\infty}\left(1 + \frac{1}{k}\right)^k
\end{equation}

\subsection{The exponential function}
% TODO plot y=e^x
The exponential function f(x) = ex, often written as exp x (see Figure 1.7) arises whenever a quantity grows or decays at a rate proportional to its size: radioactive decay, population growth and continuous interest, for example. The exponential function is defined, using the concept of a limit, as

\section{Exercises}
\begin{ExerciseList}
\end{ExerciseList}
