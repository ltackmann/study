\chapter{Introduction}
It has been said that the typical theorem in mathematics states that something you do not understand is equal to something else you cannot compute. There is a kernel of truth in this joke, since the rigor required when doing mathematical research has found is way into every nook and corner of every textbook and thereby to a large extend rendered them unreadable to all, except fairly expert mathematicians; and even these usually disdain from reading such texts cover to cover. The value of mathematical rigor is not in the question; but such works are condemned to remain works of reference, to be consulted, not read.

This is unfortunate since many people derive great happiness from dabbling with a little mathematics; just consider the many people who daily tries to solve news paper Sudoku puzzles. Such activities implies that many people enjoys the pleasures of the mind and might even find a little mathematical education fulfilling. However it would be foolish not to acknowledge the complexity of modern mathematics. After all Andrew Wiles incredible proof of Fermat's Last Theorem or Abel's remarkable proof of the in-solvability of general higher degree equations eluded many of the worlds best mathematicians for centuries. Such gems are usually only available to the person equipped with a sharp mind, a fairly developed bag of mathematical tools and a certain intellectual maturity. What I propose here is a book that will build up enough knowledge to show the proofs of some of the greatest theorems in the history of mathematics; from Hippocrates Quadrature of the Lune to Lindeman's proof that $\pi$ is transcendental. I have sought to make the text completely self contained, so the reader will be taken from simple arithmetic to measure theory and game theory.

The mathematical mature person is right to ask if such an bold attempt is not futile and perhaps its author is a bit naive. There goes a story that the great physicist Richard Feynman was once asked by a Caltech faculty member to explain why spin $1/2$ particles obey the \index{Fermi-Dirac statistics}. He nodded and said, "I'll prepare a freshman lecture on it". But a few days later he returned and said. "You know, I couldn't do it, I couldn't reduce it to freshman level. That means we really don't understand it." Just as Feynman found out this author has also had to abandon a number of otherwise interesting subjects due to the complexity of conveying them within a reasonable span of pages.

The text includes a number of exercises along with answers to every one (at least briefly). I am a firm believer in learning-by-looking, also known as cheating, and thus I encourage the reader to give each problem his best shot, then check the answer. If correct then move on to a more advanced problem, if not then study the provided solution and try to solve a similar problem yourself. If you are stuck then don't panic I have included an entire chapter devoted to the fine art of problem solving and mathematical proof.

When mathematics is thought at school its usually presented as a series of fully developed ideas and we seldom learn about their historical evolution, leaving us no clue as to how or why anyone would come up with them in the first place. However if one digs through the history of mathematics one will often see that the people we now revere as immortal masters of mathematics often struggled significantly to come to the results we teach today. Further, and much more importantly, we often discover that arcane definitions used today often where introduced to get around obstacles facing their inventors. It is my firm opinion that root problem when teaching modern mathematics is that these original developments are now largely forgotten. Today we introduce concepts such as groups and rings in algebra as if their existence was self evident. There by completely ignoring the fact that these concepts was late developments introduced in the face of trying to handle the task of finding a general solution to the quintic formula.

This book tries to make good on some of these issues by introducing mathematics in the context that faced its inventors. However the proofs presented here are not always laid out in their historical order, instead I have strived to provide the clearest proofs; so for example, instead of \index{Oresme}'s original verbal proof of the divergence of \index{harmonic series} I have included a new simple algebraic proof. At other times the original proof is so beautiful or provide a vital lessons in mathematical reasoning that I let it's author speak directly (such as Euclid's proof of the Pyteagorian theorem).
The knowledge and proofs contained herein is taken from a multitude of sources, many have been rewritten or reordered so to make them more accessible, but often one finds mathematics written with such breathtaking clarity as so to render any further attempt upon simplification impossible. In these cases the information is simply restated and put in context.

\indent Mathematics is utility and its usage is spread far and wide; from logical reason about programming languages to the physical sciences. So stories about its usages is included whenever I have found it fitting. At the same time mathematics is also history, from Archimedes war machines used against the intruding romans to everyday stories such as Newton's remark that he "do not like to be teased by foreigners about mathematical things". In the end all that I hope to achieve is to shine a little light on the fascinating field of human creativity known as mathematics, to hopefully illustrate the meaning of the the great philosopher Spinoza's words, when he said that "God is a mathematician". \\
\flushright Lars Tackmann \\ Copenhagen, Denmark\flushleft

% TODO \section{How this book is structured}

%. Equations to cover (TODO 17 equations that changed the world)
% 
% - Pythagorean theorem: helped us create better maps. We use this theorem to find the shortest distance. It is a useful technique for architecture, woodworking, or other physical construction projects.
% 
% - Logarithms (John Napier) helped us perform tedious calculations before there weren’t any calculators. They are especially evident in science and measurement. When we talk about extremely small and large numbers, we always use logarithms. For instance, when we work on our sensitivity to light, earthquake significances, noise levels in decibels, acidity (pH), money growing with a fixed interest rate, bacteria growing in a petri dish, and radioactive decay, we use logarithms.
% 
% TODO more at https://medium.com/however-mathematics/17-equations-that-changed-the-world-b2f22d2d2bdb

% Set theory
%\section{Counting the Infinite}
%\subsection{Problems with naive set theory}
%\section{Axiomatic set theory}
%\section{Cantor and the Transfinite Realm}

% Morphism In many fields of mathematics, morphism refers to a structure-preserving mapping[disambiguation needed] from one mathematical structure to another.