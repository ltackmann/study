\chapter{Mathematical foundations}

\epigraph{You just keep pushing. You just keep pushing. I made every mistake that could be made. But I just kept pushing.}{Rene Descartes}

\section{Problem solving}
\subsection{Word problems}
"A bat and ball, together, cost a total of $1.10$ and the bat costs $1$ more than the ball. How much is the ball?" The wrong answer is the one that roughly one in every two people blurts out: 10 cents. The correct answer is 5 cents, since only with a bat worth $1.05$ and a ball worth $5$ pence are both conditions satisfied.

\subsection{Problem solving checklist}

\section{Proving theorems}
% TODO http://jeremykun.com/2015/06/08/methods-of-proof-diagonalization/
\subsection{Proof by induction}
\subsection{Proof checklist}

\section{Algorithms}
A large part of mathematics deal with procedures for obtaining results such as the grade school procedure thought to millions of children for multiplying two integers, Euclids method for computing the greatest common denominator or Newtons method for computing roots. Such procedures are known as algorithms and are integral part of modern mathematics. It is the author opinion that understanding how mathematics is applied in algorithms to obtain results strengths the students understand of the subject thus a large array of algorithms are included. That being said algorithms are usually a subject for programmers and any students who only wish a firm grasp on the mathematics in this book can safely skip the algorithms or return to them later if the need should arise.

\myindent In this section we will introduce the format used to describe algorithms, sometimes known as pseudo code, that is a simplified form of programming. 

% TODO move to number theory and add proof
\begin{algorithm}
    \caption{Euclid’s algorithm}
    \label{euclid}
    \begin{algorithmic}[1] % The number tells where the line numbering should start
        \Procedure{Euclid}{$a,b$} \Comment{The g.c.d. of a and b}
            \State $r\gets a \bmod b$
            \While{$r\not=0$} \Comment{We have the answer if r is 0}
                \State $a \gets b$
                \State $b \gets r$
                \State $r \gets a \bmod b$
            \EndWhile\label{euclidendwhile}
            \State \textbf{return} $b$\Comment{The gcd is b}
        \EndProcedure
    \end{algorithmic}
\end{algorithm}

\subsection{Recursive methods}
standard refresher on how to write a recursive method:

- Are we in a trivial case? Then solve the problem and return the solution.
- Otherwise, split the problem into one or more strictly smaller problems.
- Solve the subproblems recursively.
- Combine their solutions to solve the larger problem.
