\chapter{Mathematical foundations}

% Ideas for rewrite
% Be active - Read the book. Do the exercises set. • Think for yourself - Always good advice. • Question everything - Be sceptical of all results presented to you. Don’t accept them until you are sure you believe them. • Observe - The power of Sherlock Holmes came not from his deductions but his observations. • Prepare to be wrong - You will often be told you are wrong when doing mathematics. Don’t despair; mathematics is hard, but the rewards are great. Use it to spur yourself on.
% Don’t memorize - seek to understand - It is easy to remember what you truly understand.

% quotes
% Question: How many months have 28 days? Mathematician’s answer: All of them.- 

% Reading mathematics
% - excercies

% Writing mathematics
% - excercies
% - words or symbols
% - clarity

% How to solve problems
% - Polya for step plan
% - understanding the problem, devising a plan, executing a plan, looking back
% - Find a simpler problem you can solve

% How to think logically
% - statement
% 	- negation
% 	- truth tables
% 	- equivalence
% - Implications
%   - Hypothesis, asumptions and conclusion
% 	- A implies B
% 	- Common mistakes on the inverse
% - Converse and equivalence
% - For all there exists
% - Examples and counter examples

% Definitions theorems and proofs
% - Definitions and how to read them
% - How to read a theorem
% - Proofs
%	- Why so hard to read

% Proof tequniques
% - Direct method
% 	- If and only if proofs
%	- proving one set is a subset of another
% 	- proving that two sets are equal
% 	- Common mistates
% Proof by case
% Proof by contradiction
% Proof by induction
% - advanced induction techniques
% Contrapositive method

% Basic mathematics
% - Divisors
% 	- Infinitude of primes
%	- GCD
% 	- Euclids algorithm
% 	- Diophantine equations
% - Modular arithmetic
% - Ijective, surjective, bijective
% - Equivalence relations
% 	- equivalance classes

% Generalizsation and specialisation

\epigraph{You just keep pushing. You just keep pushing. I made every mistake that could be made. But I just kept pushing.}{Rene Descartes}

\section{Problem solving}
\subsection{Word problems}
"A bat and ball, together, cost a total of $1.10$ and the bat costs $1$ more than the ball. How much is the ball?" The wrong answer is the one that roughly one in every two people blurts out: 10 cents. The correct answer is 5 cents, since only with a bat worth $1.05$ and a ball worth $5$ pence are both conditions satisfied.

\subsection{Problem solving checklist}

\section{Proving theorems}
% TODO http://jeremykun.com/2015/06/08/methods-of-proof-diagonalization/
\subsection{Proof by induction}
\subsection{Proof checklist}

\section{Algorithms}
A large part of mathematics deal with procedures for obtaining results such as the grade school procedure thought to millions of children for multiplying two integers, Euclids method for computing the greatest common denominator or Newtons method for computing roots. Such procedures are known as algorithms and are integral part of modern mathematics. It is the author opinion that understanding how mathematics is applied in algorithms to obtain results strengths the students understand of the subject thus a large array of algorithms are included. That being said algorithms are usually a subject for programmers and any students who only wish a firm grasp on the mathematics in this book can safely skip the algorithms or return to them later if the need should arise.

\myindent In this section we will introduce the format used to describe algorithms, sometimes known as pseudo code, that is a simplified form of programming. 

% TODO move to number theory and add proof
\begin{algorithm}
    \caption{Euclid’s algorithm}
    \label{euclid}
    \begin{algorithmic}[1] % The number tells where the line numbering should start
        \Procedure{Euclid}{$a,b$} \Comment{The g.c.d. of a and b}
            \State $r\gets a \bmod b$
            \While{$r\not=0$} \Comment{We have the answer if r is 0}
                \State $a \gets b$
                \State $b \gets r$
                \State $r \gets a \bmod b$
            \EndWhile\label{euclidendwhile}
            \State \textbf{return} $b$\Comment{The gcd is b}
        \EndProcedure
    \end{algorithmic}
\end{algorithm}

\subsection{Recursive methods}
standard refresher on how to write a recursive method:

- Are we in a trivial case? Then solve the problem and return the solution.
- Otherwise, split the problem into one or more strictly smaller problems.
- Solve the subproblems recursively.
- Combine their solutions to solve the larger problem.
