\chapter{Set theory}
% TODO Sets missing concepts
% - union, complement and intersection of sets
% - filtered set, unordered sets, ordered set
%\section{Naive set theory}
%\section{Counting the Infinite}
%\subsection{Problems with naive set theory}
%\section{Axiomatic set theory}
%\section{Cantor and the Transfinite Realm}
% Morphism In many fields of mathematics, morphism refers to a structure-preserving mapping[disambiguation needed] from one mathematical structure to another.

\section{Sets}
A indexed set is a collection of values associated with indices. For example
\begin{itemize}
\item An ordered pair is a family indexed by the two element set $2 = \{1, 2\}$.
\item An n-tuple is a family indexed by $n$.
\end{itemize}
the set that whose members label (or index) members of a family is called an index set. Some common operations on sets
\begin{description}
\item[Empty set] $\varnothing = \{\}$.
\item[Set intersection] $S \cap T$ is the set containing all the elements that are in both $S$ and $T$:
\begin{equation}
S \cap T := \left\{{x: x \in S \land x \in T}\right\}
\end{equation}
\item[Set union] $S \cup T$ is the set containing all the elements that are in either or both of the sets $S$ and $T$:
\begin{equation}
S \cup T := \left\{{x: x \in S \lor x \in T}\right\}
\end{equation}
\end{description}

\section{Exercises}
\begin{ExerciseList}
\end{ExerciseList}
