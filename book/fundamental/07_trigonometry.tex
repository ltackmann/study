\chapter{Trigonometry}

the  last  great  creation  of  the  Greek  period,  plane  and
 spherical  trigonometry  by  Hipparchus  and  Ptolemy  and  their  application  to
 one  of  the  dreams  of  mankind,  understanding  the  movements  of  the  heavenl bodies.  This  gave  rise  to  modern  astronomy  and  the  physical  sciences.


% TODO missing trig information
% - add figures for and explain why sin, cos and tan exists
% - proove the rrlations between the trig functions

Trigonometric functions are functions of angles. Figure 1.2 shows a right-angled
triangle (ie a triangle with one angle of 90°) with another angle denoted by
the Greek letter theta θ. The sides of the a right-angled triangle are called
the adjacent side (next to θ), the opposite side (opposite to θ) and the
hypotenuse (opposite the right-angle). For any right-angled triangle, the
ratios of the various sides are constant for any particular value of θ. The
basic trigonometric functions are:

\subsection{Trigonometric relations}
\begin{equation}
    tan(\theta) = \frac{sin(\theta)}{cos(\theta)}
\end{equation}

The reciprocals of the $cosine$, $sine$, and $tangent$ are known as
$secant$ ($sec$), $cosecant$ ($csc$), and $cotangent$ ($cot$):
\begin{equation}
    sec(\theta) = \frac{1}{sin(\theta)}
\end{equation}
\begin{equation}
    csc(\theta) = \frac{1}{cos(\theta)}
\end{equation}
\begin{equation}
    cot(\theta) = \frac{1}{tan(\theta)}
\end{equation}

\subsection{Inverse trigonometric functions}

\section{Exercises}
\begin{ExerciseList}
\end{ExerciseList}
