\chapter{Linear algebra}
% TODO missing linear algebra concepts
% - http://resources.codingthematrix.com
% - http://en.m.wikipedia.org/wiki/Perron–Frobenius_theorem
% - spectral theorem https://inst.eecs.berkeley.edu/~ee127a/book/login/l_sym_sed.html
% - discrete fourier transform http://jeremykun.com/2012/06/23/the-discrete-fourier-transform/
% - fast fourier transform http://jeremykun.com/2012/07/18/the-fast-fourier-transform/
% - http://www.physicsclassroom.com/class/vectors/Lesson-1/Vector-Addition

% field
Real numbers \\
Complex numbers\\
Finie field GF2

% vectors
Vectors \\
Vector addition
Vector scalar multiplication
Convex combinations
Affine combinations
Dot product 
Solving triangular systems

% vector space
Linear combination
Span
Vector spaces
Afine spaces
Linear systems, homogenous and otherwise

% Matrix
Rows, columns and entries
identity matrix
Column space and row space
Matrix vector and vector matrix multiplication
 - dot products
 - linear combinations
Null space
Sparse matrix
Linear functions
Matrix matrix multiplication
Column vector 
Row vector
inner product 
outer product
Matrix inverse

% The basis
Minimum spanning forest

% Dimension
Dimension and rank

% Gausian eleminiation
Echelon form
Solving vector matrix equations
Factoring integers

% The inner product
Distance, length, norm and inner product
Orthogonality

% Orthogonalization
QR factorisation
Least squares
Line fitting

% Special bases
Wavelets
Fourier transform
Discrete foruer transform

% Singular value decomposition
Low rank matrixes

% The eigenvector
Discrete dynamic processes
Diagonalisation and the Fibonacci matrix
Eigen values and eighen vectors
Markov chains

% Linear programming
Geometry of linear programming
The simplex algorithm
Game theory


\[ 
\norm{a \vec{u}} = \abs{a} \, \norm{\vec{v}} 
\]

\section{Vectors}
numbers. Another example: when representing, say, a  1024×768 black-and-white image as a vector, we define the vector as a function from the domain $D = {1,..., 1024} × {1, .., 768}$ to the real numbers. The function specifies, for each pixel $(i, j)$, the image intensity of that pixel. 

% vector dot product
\begin{description}
    \item[Addition]
    \item[Inner product (dot product)]
    \item[Pointwise vector division]
    \item[Pointwise vector multiplication]
    \item[Saxpy]
    \item[Scalar vector multiplication]
\end{description}

\section{Matrices}
% http://www.eng.buffalo.edu/Research/BEST/Research/Lecture%20Series%202013/Matrix%20Multiplication.pdf
% transpose http://www.mathwords.com/t/transpose_of_a_matrix.htm
% inverse http://www.mathwords.com/i/inverse_of_a_matrix.htm
\begin{description}
    \item[Addition]
    \item[Pointwise matrix division] (denominator matrix must have non-zero entries)
    \item[Pointwise matrix multiplication]
    \item[Rank] The rank of a matrix is the number of linearly independent columns (which is equal to the number of linear independent rows)
    \item[Scalar matrix multiplication]
\end{description}

\section{Exercises}
\begin{ExerciseList}
\end{ExerciseList}
