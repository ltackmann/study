\chapter{Linear algebra}
% TODO missing linear algebra concepts
% - http://resources.codingthematrix.com
% - http://en.m.wikipedia.org/wiki/Perron–Frobenius_theorem
% - spectral theorem https://inst.eecs.berkeley.edu/~ee127a/book/login/l_sym_sed.html
% - discrete fourier transform http://jeremykun.com/2012/06/23/the-discrete-fourier-transform/
% - fast fourier transform http://jeremykun.com/2012/07/18/the-fast-fourier-transform/
% - http://www.physicsclassroom.com/class/vectors/Lesson-1/Vector-Addition





\[ 
\norm{a \vec{u}} = \abs{a} \, \norm{\vec{v}} 
\]

\section{Vectors}
% vector dot product
\begin{description}
    \item[Addition]
    \item[Inner product (dot product)]
    \item[Pointwise vector division]
    \item[Pointwise vector multiplication]
    \item[Saxpy]
    \item[Scalar vector multiplication]
\end{description}

\section{Matrices}
% http://www.eng.buffalo.edu/Research/BEST/Research/Lecture%20Series%202013/Matrix%20Multiplication.pdf
% transpose http://www.mathwords.com/t/transpose_of_a_matrix.htm
% inverse http://www.mathwords.com/i/inverse_of_a_matrix.htm
\begin{description}
    \item[Addition]
    \item[Pointwise matrix division] (denominator matrix must have non-zero entries)
    \item[Pointwise matrix multiplication]
    \item[Rank] The rank of a matrix is the number of linearly independent columns (which is equal to the number of linear independent rows)
    \item[Scalar matrix multiplication]
\end{description}

\section{Exercises}
\begin{ExerciseList}
\end{ExerciseList}
