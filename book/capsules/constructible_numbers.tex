\begin{framed}
\textbf{Constructible Numbers}\\ A number is said to be constructible if it can be represented by a finite number of additions, subtractions, multiplications, divisions, and square root extractions of integers. Such numbers correspond to line segments which can be constructed by beginning with a unit length (that is, a length to represent the number "$1$") and keep track of what  other lengths we can produce by straightedge and compass construction.

\myindent It turns out that the totality of all possible constructible lengths, while vast, does not include every real number; as it can be shown that all constructible numbers are algebraic it follows that no transcendental numbers can be constructed. However integers such as $1,2,3$ and rationales such as $1/2, 1/3, 1/4$ and square roots, like $\sqrt{2}$ and $\sqrt{5}$ can all be constructed. Further, if we can construct two  magnitudes, we can construct their sums, difference, product and quotient and thus even complex expressions such as
\[
\sqrt{\frac{2+\frac{1}{13}\sqrt{3}}{1 + \sqrt{4 + \sqrt{3}}}}
\]
are actually constructible.
\end{framed}
