\chapter{Greek Geometry}
\section{Hippocrates' Quadrature of the Lune}
\section{Euclidian Geometry}
\begin{prop}{\textbf{III.21}}
In a circle, angles in the same segment are equal to one another. 
\end{prop}
\begin{proof} 
Let ABCD be a circle, and let the angles BAD, BED be angles in the same segment BAED; I say that the angles BAD,BED are equal to one another.

For let the center of the circle ABCD be taken, and let it be F; let BF, FD be joined.

Now, since the angle BFD is at the center, and the angle BAD at the circumference, and they have the same circumference BCD as base, therefore the angle BFD is double the the angle BAD. [III. 20] For the same reason the angle BFD is also double the angle BED.

Therefore the angle BAD is equal to the angle BED.

Therefore in a circle the angles in the same segment are equal to one another. $\qedhere$
\end{proof}

\section{Spheres, cylinders and Archimedes}
\section{Euclid in the 21st century}

\chapter{Word Problems (Ramsey Theory)}
- "There are six people in the classroom. Prove that among them  there must be either three people who do not know one another or  three people who all know one another."

- how many people must be invited to a party so that at  least m will know one another or at least n will not know one another.

- How many different   patterns can you create with thread while sewing on a four-hole   button?

\chapter{Topology}
The best example of a manifold is the Earth as portrayed   through a series of maps, each showing only a small part of  its surface. Imagine a map of Manhattan, for example: its Euclidean   nature is obvious. When maps are put together in an atlas,  their parallel lines continue not to cross and their triangles maintain   their 18o-degree nature.

A rubber band can be slipped off any place on a ball, a box, a bun,  or a blob without a hole, which makes them all essentially similar  or, in the language of topology, diffeomorphic to one another. This  means you can reshape any one of them into any other and then  back again.

\chapter{Statistics}
Regression is the main statistical technique used to quantify the relationship between two or more variables. It was invented by
Adrien-Marie Legendre in 1805. A regression analysis would show a positive relationship between height and weight, for example.
If we threw in waistline along with height, we’d get an even better regression to predict weight. The measure of the accuracy of a
regression is called R-squared. A perfect relationship, with no error, would have an R-squared of 1.00 or 100

Strong relationships, like height and weight, would have an R-squared of around 70 percent. A meaningless relationship, like zip
code and weight, would have an R-squared of zero.

The way probability works is that the likelihood of a series of independent events occurring is the multiplication of the
likelihood of each individual event (also known as compound probability). So, if the probability of you finishing this chapter is
9 out of 10 (9/10), and the probability of you finishing the next one is 9/10, the total probability of you finishing both
chapters isn't 9/10: it's 81/100.

%%% EUCLIDIAN GEOMETRY
%%% http://math.furman.edu/~jpoole/euclidselements/eubk3/props.htm
