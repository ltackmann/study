\chapter{Curious and Interesting Numbers}
\section{Numerous Curiosities}
\subsection{Perfect Numbers} 
A number which is the sum of all its digits is said to be perfect, (Euclid IX.36)

\section{Fibonacci Numbers}
The Fibonacci sequence of numbers is a pleasantly simple class of numbers and perhaps the mathematical device most 
frequently used in nature. The first few numbers are
\begin{center}
\begin{tabular}{c|*{11}c}
$n$      & $0$ &  $1$   & $2$ &  $3$ & $4$ & $5$ & $6$ & $7$  & $8$    & $9$   & $10$ \\
\hline
$F_n$ & $0$ &  $1$   & $1$ &  $2$ & $3$ & $5$ & $8$ & $13$ & $21$ & $34$ & $55$ 
\end{tabular}
\end{center}

They are defined by the requrence:
\[\begin{array}{lcl} 
F_0 & = & 0 \\
F_1 & = & 1 \\
F_n & = & F_{n-1} + F_{n-2}
\end{array}\]
a simple recurence which simply ensures that every Fibonacci number is the sum of the two previous ones.

Fibonacci numbers have some intersting propertie, observing the list above one is tempted to conclude that every 3rd
number is even:
\begin{proof}

\end{proof}

inspecting the sum of the first $n$ Fibonacci numbers suggests TODO
\begin{proof}
We will attempt a proof of $\sum_{k=1}^{n} = F_{n+2} - 1$ by induction
\end{proof}

\myindent One thing you quickly release when computing Fibonacci numbers is that there is a whole lot of computation going on, due
to the repeatet double recursive calls. A tail-recursive function is a function which only have one recursive call, this
property makes them easier to handle and often faster to compute so lets attempt to remove one of the recursive calls


\subsection{Generating Functions}
One might ask if we can perhaps remove these calls and compute the numbers directly. 
