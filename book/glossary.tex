% use \gls{} or \Gls to reference singular names
% use \glspl or \Glspl to reference plural names

% TODO missing from glossary
% Domain and CoDomain
% Sujective, Bijective and injective
% Monomial, binomial, polynomial
% Quadratic equations
% Product, Factors
% DC
% DB
%* Cashflow
%* PV
%* PMT
%* FV
%* Principle (capital amount)
% - the amount lent or borrowed is usually refered to as the principle or capital amount
%* **Market value:**
%* **Provision (hensættelse):** an account which records a present liability of the plan provider to its policy holders.
% coupon rate
% bond equivalent yield
% SPV (stochastic present value)
% ABO (accumulated benefit obligation)
% PBO (projected benefit obligation)
% Annuity
% LumpSum
% Brownian motion
% Gompertz-Makeham mortality
% Put options
% Utility function of wealth
% Bernoulli random variable
% Manifolds

\glsentry{associativity}{operation is associativ if within
an expression containing two or more occurrences in a row of the same
associative operator, the order in which the operations are performed
does not matter as long as the sequence of the operands is not changed.}

\glsentry{axiom}{}

\glsentry{cohort}{A group of subjects who have shared a particular
event together during a particular time span (e.g., people born in Europe
between 1918 and 1939; survivors of an aircrash; truck drivers who smoked
between age 30 and 40)}

\glsentry{commutativity}{Commutativity operation is commutative if
changing the order of the operands does not change the result.}

\glsentry{congruent}{Two figures are congruent if they have the same
shape and size, but are in different positions}

\glsentry{constant}{a number, term or expression that doesn't change. If
$y = x + 7$ then 7 is a constant. The area of a circle equals $2\pi r$ where
$r$ is the radius (a variable), and $\pi$ is a constant.}

\glsentry{cube}{the cube of a number is the number multiplied by itself
twice, eg 3 cubed = $3^3 = 3 × 3 × 3 = 27$.}

\glsentry{denominator}{}

\glsentry{factor}{a number or expression that divides exactly into
another number or expression. 1,2 and 5 are factors of 10; x and (2x + 3)
are factors of 2x2 + 3x because x multiplied by (2x + 3) = 2x2 + 3x.}

\glsentry{fraction}{a number such as 1/2, 2/3, 17/2. Can also be written
with a horizontal line, eg . The number above the line (called the
numerator) is divided by the number below the line (called the
denominator).}

\glsentry{frustrum}{A solid, usually a pyramid or a cone, with its top
cut of by a plane, parallel to its base.}

\glsentry{integer}{Positive and negative whole numbers such as -34, -5,
0, 1, 17, 1021}

\glsentry{irrational number}{A number that is not rational, eg $\pi$,
Euler's number e.}

\glsentry{isosceles triangle}{Isosceles triangle is a triangle that has
two sides of equal length.}

\glsentry{linear equation}{A algebraic expression, usually of the form
\[
ax = b
\]
in which each term is either a constant or the product of a constant and
(the first power of) a single variable (that is no variables in the
expression contains exponents, like the $2$ in $x^2$, square roots,
cube roots, etc.)}

\glsentry{measure}{A measure on a set is a systematic way to assign a
number to each suitable subset of that set, intuitively interpreted as
its size. In this sense, a measure is a generalisation of the concepts
of length, area, and volume}

\glsentry{numerator}{}

\glsentryref{postulate}{axiom}

\glsentry{product}{The result of multiplying together two or more numbers
or things, eg the product of 4 and 5 is 20.}

\glsentry{QED}{\emph{Quod erat demonstrandum}, Latin for
\emph{which was to be proved}}

\glsentry{quadratic polynomial}{A polynomial with
degree 2 such as $7x^2 + 4x - 10$.}

\glsentry{quotient}{The result of dividing one number by another. The
quotient of 14 and 7 is 2 as 14/7 = 2.}

\glsentry{rational expressions}{A expression
involving fractions of polynomials such as $\frac{x-1}{x^2 + 12}$.}

\glsentry{rational number}{A number that can be made by dividing one
integer by another, eg 1/2, 5/7, 12/108, 12/1.}

\glsentry{real number}{the normal numbers we use, including decimals,
fractions, integers, negative numbers, positive numbers, irrational
numbers, etc. They are called real because they aren't complex numbers
(numbers that involve the square root of -1)}

\glsentry{similarity}{Two geometrical objects are called similar if they
both have the same shape, that is one can be obtained from the other by
uniformly scaling (enlarging or shrinking)}

\glsentry{square}{the square of a number is the number multiplied by
itself, eg 5 squared = 52 = 25}

\glsentry{square root}{the square root of a number is a number that when
multiplied by itself gives the original number. The square root of 36 is
6. The square root symbol is , so .}

\glsentry{statement}{\emph{If P, then Q}, where its
\textbf{converse} has the following form: \emph{if Q, then P}}

\glsentry{term}{}

\glsentry{variable}{a quantity that may change, eg the circumference of a
circle equals 2πr where π (pi) is a constant and r, the radius, is a
variable.}
