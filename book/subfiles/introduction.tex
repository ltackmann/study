\chapter{Introduction}
It has been said that the typical theorem in mathematics states that something you do not understand is equal to something else you cannot compute. There is a kernel of truth in this joke, since the rigor required when doing mathematical research has found is way into every nook and corner of every textbook and thereby to a large extend rendered them unreadable to all, except fairly expert mathematicians; and even these usually disdain from reading such texts cover to cover. The value of mathematical rigor is not in the question; but such works are condemned to remain works of reference, to be consulted, not read.

This is unfortunate since many people derive great happiness from dabbling with a little mathematics; just consider the many people who vigorously engages the daily news paper Sudoku puzzles. Such spending of time implies that many people indeed still enjoys the pleasures of the mind. However it would be foolish not to acknowledge the complexity of modern mathematical developments. After all Andrew Wiles incredible proof of Fermat's Last Theorem or Abel's remarkable proof of the in-solvability of general higher degree equations eluded many of the worlds best mathematicians for centuries. Such gems are usually only available to the person equipped with a sharp mind, a fairly developed bag of mathematical tools and a certain intellectual maturity. What I propose here is a book that will build up enough knowledge as to show the proofs of the greatest theorems in the history of mathematics; from Hippocrates Quadrature of the Lune to Lindeman's proof that Pi is transcendental. I have sought to make the text completely self contained, so the reader will be taken from simple arithmetic to measure theory and game theory.

The mathematical mature person is right to ask if such an bold attempt is not futile and that perhaps this author is a bit naive. There goes a story that the great physicist Richard Feynman was once asked by a Caltech faculty member to explain why spin $1/2$ particles obey the \index{Fermi-Dirac statistics}. He nodded and said, "I'll prepare a freshman lecture on it". But a few days later he returned and said. "You know, I couldn't do it, I couldn't reduce it to freshman level. That means we really don't understand it." Just as Feynman found in his great lectures on physics this author has also had to abandon a number of otherwise interesting subjects due to the complexity of conveying them.

The text includes a number of exercises along with answers to every one (at least briefly). I am a firm believer in learning-by-looking, also known as cheating, and thus I encourage the reader to give each problem his best shot, then check the answer. If correct then move on to a more advanced problem, if not then study the provided solution and try to solve a similar problem. If you are stuck then don't panic I have included an entire chapter devoted to the fine art of problem solving and mathematical proof.

The subject presented here is not laid out in the usual chronological manner, instead I have sought to build up mathematics as a logical progression of ever more complex ideas. Therefor instead of focusing on historical order I have strived instead to provide the clearest proofs; so for example, instead of \index{Oresme}'s original verbal proof of the divergence of the Harmonic Series I have instead included a never and simpler algebraic proof. At other times the original proof is so beautiful or provide a vital lessons in mathematical reasoning that I let it's author speak directly (such as Euclid's proof of the Pyteagorian theorem).

The knowledge and proofs contained herein is taken from a multitude of sources, many have been rewritten or reordered so to make them more accessible, but often one finds mathematics written with such breathtaking clarity as so to render any further attempt upon simplification impossible. In these cases the information is simply restated and put in context. In the end of each chapter propper credit is provided as well as suggestions on further reading for the adventures person.\\

\indent Mathematics is utility and its usage is spread far and wide; from logical reason about programming languages to the physical sciences. So stories about its usages is included whenever I have found it fitting. At the same time mathematics is also history, from Archimedes war machines used against the intruding romans to everyday stories such as Newton's remark that he "do not like to be teased by foreigners about mathematical things". In the end all that I hope to achieve is to shine a little light on the fascinating field of human creativity known as mathematics, to hopefully illustrate the meaning of the the great philosopher Spinoza's words, when he said that "God is a mathematician". \\
\flushright Lars Tackmann \\ Copenhagen, Denmark\flushleft


% Quotes
