\chapter{Actuarial mathematics}
Until the beginning of the 1980s, the concept of a pension was largely
synonymous with the idea of a defined benefit (DB) plan; that is, the promise
of a pension meant that upon retirement, an individual would receive a
predetermined amount for the rest of his or her life.

Beginning in the 1980s, there was substantial growth in defined contribution
(DC) plans. In these plans, the employer does not specify what benefits the
retiree will receive. Instead, an investment account is opened for an employee,
the employer and/or the employee make contributions to it, and upon retirement
the individual can draw upon the money in the account.

Depending on how much is contributed and how well it is invested, the retiree
could have more than enough for a comfortable retirement, or less than enough.

Assume you enter the labor force at the age of 35. Your job pays a fixed 50,000
per year for the next thirty years, after which you retire at age 65. This job
provides no pension instead its your responsibility to ensure you save enough
to maintain a dignified standard of living during your retirement. Finally also
ignore taxes (both income tax and any possible deductiones earned from
participating in a pension plan) and we assume you die at age 95.
