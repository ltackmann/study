\chapter{Calculus}
% TODO missing calculus concepts
% http://www.science4all.org/le-nguyen-hoang/infinite-series/
% - Area under curve
% - Fundamental theorem of calculus
% - Newton rapson method for square roots http://www.sosmath.com/calculus/diff/der07/der07.html
% - Proof and explain taylor series
% - Proof Finite geometric series https://www.khanacademy.org/math/precalculus/seq_induction/geometric-sequence-series/v/geometric-series-introduction
% - Numerical integration http://rosettacode.org/wiki/Numerical_integration#ActionScript
% - LaplaceTransform http://reference.wolfram.com/language/ref/LaplaceTransform.html
% - story of e (proof the limit of the series)
% - relationship between pi, e and i
% - include info on e from "the story of e"
% - include info on i from  "the story of i"
% Gamma function and Stirling's approximation
% TODO eulers identity

\section{Series}

\section{Logarithm}
A logarithm (log or logs for short) goes in the opposite direction to a power by asking the question: what power produced this number? So, if 2x = 32, we are asking, ‘what is the logarithm of 32 to base 2?’ We know the answer: it's 5, because 25 = 32, so we say the logarithm of 32 to base 2 is 5. In general terms, if , then we say the logarithm of a to base x equals p, or For example, 104 = 10,000, so we say the logarithm of 10,000 to base 10 equals 4, or We can take logarithms of any positive number, not just whole ones. So, as 103.4321 = 2704.581 we say the logarithm of 2704.581 to base 10 equals 3.4321, or Logarithms to base 10 are called common logarithms. Older readers may remember, many years ago - after the dinosaurs, but before calculators and computers were widely available - doing numerical calculations laboriously by hand using tables of common logarithms and anti-logarithms. The properties of logarithms are based on the aforementioned rules for working with powers. Assuming that a > 0 and b > 0 we can say: logx(ab) = logxa + logxb, eg log10(1000 × 100) = log10(100,000) = 5 = log10(1000) + log10(100) = 3 + 2. logx(1/a) = -logxa, eg log3(1/27) = -log327 = -3. logx(a/b) = logxa - logxb, eg log2(128/8) = log2(16) = 4 = log2(128) - log2(8) = 7 - 3. logx(ay) = ylogxa, eg log5(253) = log515,625 = 6 = 3 × log525 = 3 × 2. 1.8.3

% TODO algorithms for calculating logarithms (Taylor series) http://en.wikipedia.org/wiki/Logarithm#Power_series

\subsection{Natural logarithm}
% TODO plot y = ln{x}
Say we invest £1.00 in an exceedingly generous bank that pays 100\% interest per annum. If the bank calculated and credited the interest at the end of one year, our investment would then be worth 1 + 1 = £2.00. But what if the bank credits the interest more frequently than once a year? If interest is calculated and added every six months, at the end of that period the balance would equal and at the end of one year the total amount would be . Calculated three times a year, the final balance would be . In general, if interest is calculated n times a year, the balance x after one year is
% TODO use correct symbols as used in compound intersts 
% TODO add as compound ingerest in index
\[
f(k) = \left(1 + \frac{1}{k}\right)^k
\]
if we plug in some values we observe
\[
\begin{align*}
  f(2)  =& \left(1 + \frac{1}{2}\right)^2 = 2.25 \\
  f(5)  =& \left(1 + \frac{1}{5}\right)^5 = 2.248832 \\
  f(10) =& \left(1 + \frac{1}{10}\right)^{10} = 2.59374...
\end{align*}
\]

% TODO place in table
n 1 2 2 2.25 3 2.37037 4 2.44141 5 2.48832 10 2.59374 100 2.70481 1000 2.71692 100,000 2.71827 1,000,000 2.71828 10,000,000 2.71828

We can see that as n increases, the value of the function appears to settle down to a number approximately equal to 2.71828. It can be shown that as n becomes infinitely large, it does indeed equal the constant e. The mathematically succinct way of saying this introduces the important idea of a limit and we say where means the limit of what follows (ie ) as n approaches infinity (symbol ). In other words, e approaches the value of as n approaches infinity.

\begin{equation}
e = \lim_{k\to\infty}\left(1 + \frac{1}{k}\right)^k
\end{equation}

\section{The exponential function}
% TODO plot y=e^x
The exponential function f(x) = ex, often written as exp x (see Figure 1.7) arises whenever a quantity grows or decays at a rate proportional to its size: radioactive decay, population growth and continuous interest, for example. The exponential function is defined, using the concept of a limit, as

\subsection{Taylor series}
\begin{equation}
cos(\theta) = \sum_{n=0}^{\infty} \frac{(-1)^n \cdot \theta^{2n}}{2n!}
\end{equation}

\subsection{Geometric Series}
\begin{equation}
a + ar + ar^2 + ar^3 + \cdots + ar^{n-1} = \sum_{k=0}^{n-1}ar^k = a \frac{1-r^n}{1-r}
\end{equation}


\section{Integration}

\subsection{Laplace Transform}

\section{Numerical integration}

\subsection{Trapez}
